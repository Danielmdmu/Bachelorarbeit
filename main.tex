%===========================================================
%== Präambel ===============================================
%===========================================================
\documentclass[
    paper=a4,
    bibtotocnumbered,
    liststotocnumbered,
    oneside,
    12pt,
    listof=totoc,
    toc=chapterentrywithdots,
    listof=entryprefix,
]{scrreprt}

\usepackage[a4paper, left=2.5cm, right=2.5cm, top=1.25cm, bottom=1.25cm, includehead, includefoot]{geometry}


\usepackage{float}
\usepackage[T1]{fontenc}%Encoding
\usepackage[german]{babel}%German-specific commmands
%\usepackage[utf8]{inputenc} %Wird nicht benötigt bei XeLatex
\usepackage{fancyhdr}%% Kopfzeile u. Fußzeile
\usepackage{fontspec}%% Schriftart
\usepackage[onehalfspacing]{setspace}%-- Zeilenabstand 1,5
\usepackage[toc, nopostdot, acronyms]{glossaries}%% Glossar
\usepackage[intoc]{nomencl}%% Symbolverzeichnis
\usepackage{etoolbox}%Sorgt dafür, das Symbolverzeichnis im Header steht
\usepackage[]{minted}
\usepackage[titles]{tocloft}
\usepackage{scrhack}
\usepackage{colortbl}%Einfärben von Tabellen-Zellen, Zeilen, Spalten
\usepackage[]{layout}
\usepackage[]{blindtext}
%\usepackage{showframe}% zum Anzeigen des Seitenlayouts
\usepackage[figurename={Abb.},tablename={Tab.}]{caption}%Anpassung der Namen für Abbildung und Tabelle
\usepackage[]{url}
\usepackage[]{hyperref}
\usepackage{chngcntr}%% Nummerierung der Abbildungen und Tabellen fortlaufend %%%%%%
\usepackage{tocbibind}
\usepackage[style=alphabetic, citestyle=alphabetic, ,maxcitenames=1,uniquelist=false]{biblatex}
\usepackage[babel,german=quotes]{csquotes} % Deutsche
\usepackage{booktabs}
\usepackage[final]{showkeys}
\usepackage{tabularx}

% Paragraph styles
\setlength{\parindent}{0cm}
\setlength{\parskip}{6pt}

%% Kopfzeile u. Fußzeile
\pagestyle{fancy}
\fancyfoot{}
\fancyfoot[R]{\thepage}
\fancyhead{}
\fancyhead[L]{\nouppercase{\leftmark}}
\fancypagestyle{plain}{\pagestyle{fancy}}


%-- Standardschriftart auf Times New Roman umstellen
\setmainfont[ Path = Fonts/,
BoldFont=timesbd.ttf,
ItalicFont=timesi.ttf,
BoldItalicFont=timesbi.ttf
]{times.ttf}

%-- Kapitelüberschriften auf Standardschriftart(Times New Roman) umstellen
\setkomafont{disposition}{\normalcolor\bfseries}

% Entfernt "Kapitel X" aus der Kopfzeile vor der Kapitelüberschrift
\renewcommand{\chaptermark}[1]{\markboth{\MakeUppercase{#1}}{}}
 
%Abstände nach Kapiteln 
\RedeclareSectionCommand[%
  beforeskip=0pt,
  afterskip=1\baselineskip plus .1\baselineskip minus .167\baselineskip
]{chapter}

%% Nummerierung der Abbildungen und Tabellen fortlaufend %%%%%%
\counterwithout{figure}{chapter}
\counterwithout{table}{chapter}

%% Rahmen in Textbreite um Bilder
\floatstyle{boxed}
\restylefloat{figure}
%\restylefloat{table}
\restylefloat{listing}

%% Captions linksbündig orientieren
\captionsetup{justification=raggedright,singlelinecheck=false}

% Einbinden von Glossar u. Abkürzungsverzeichnis
%-----------------------------------------------------------------------------
% Glossar
%-----------------------------------------------------------------------------

% \newglossaryentry{Literal}
% {
%     name=Literal,
%     description={\todo{Was ist ein Literal}}
% }
\newacronym{IDE}{IDE}{Integrated Development Environment - Integrierte Entwicklungsumgebung}
 
\newacronym{FAQ}{FAQ}{Frequently Asked Questions}

\newacronym{GCD}{GCD}{Grand Central Dispatch}
\makeglossaries

% Hinzufügen von Bib-Dateien
\addbibresource{Literatur/literatur.bib}

\newminted[GoCode]{go}{linenos=true, xleftmargin=.7cm}
\newminted[SwiftCode]{swift}{linenos=true, xleftmargin=.7cm, escapeinside=@@}
\newmintedfile[InputGo]{go}{linenos=true, xleftmargin=.7cm}
\newmintedfile[InputSwift]{swift}{linenos=true, xleftmargin=.7cm}

\usepackage{color}
\newcommand{\todo}[1]{\textcolor{white}{\colorbox{red}{ To do %
      :}}\textcolor{red}{\ \ #1
  }\textcolor{red}{\colorbox{red}{III}}\ }


\def\listingautorefname{Codebeispiel}% 

\newcolumntype{L}[1]{>{\raggedright\arraybackslash}p{#1}} % linksbündig mit Breitenangabe
\newcolumntype{C}[1]{>{\centering\arraybackslash}p{#1}} % zentriert mit Breitenangabe
\newcolumntype{R}[1]{>{\raggedleft\arraybackslash}p{#1}} % rechtsbündig mit Breitenangabe


\newcommand{\dummyfig}[1]{
  \centering
  \fbox{
    \begin{minipage}[c][0.33\textheight][c]{0.5\textwidth}
      \centering{#1}
    \end{minipage}
  }
}


\begin{document}

%===========================================================
%== Titelseite =============================================
%===========================================================
\newgeometry{left=2.5cm, right=2.5cm, top=2.5cm, bottom=2.5cm}
\begin{titlepage}
    \begin{center}
    \includegraphics[width=\textwidth]{Images/logo}
        \begin{Large}
        Hochschule für angewandte Wissenschaften Coburg
        \\
        Fakultät Elektrotechnik und Informatik
        \par
        \end{Large}
        \vspace{2.0cm}
        
        \begin{Large}
            Studiengang: Informatik
            \par
        \end{Large}
        \vspace{1.5cm}
        
        {\Large
            Bachelorarbeit
        }
        \vspace{2.0cm}

        \begin{huge}
            \textbf{Vergleich von Google Go und Apple Swift} 
            \par
        \end{huge}        
        \vfill
        
        \begin{huge}
            Daniel Müller
        \end{huge}
        \vspace{2.0cm}
        
        \begin{large}
            Abgabe der Arbeit: 07. Juni 2017
            
            Betreut durch:
            \\
            Prof. Volkhard Pfeiffer, Hochschule Coburg
            \\
            %<optional: Zweitgutachter: Prof. Dr. XXX, Hochschule Coburg>
            \par
        \end{large}
        
    \end{center}
\end{titlepage}
\restoregeometry
    
%===========================================================
%== Inhaltsverzeichnis =====================================
%===========================================================   
\begin{singlespace}
\renewcommand{\cftchapleader}{\cftdotfill{\cftdotsep}} % Punkte im Inhaltsverzeichnis bei Kapiteln
\tableofcontents
\end{singlespace}

%===========================================================
%== Abbildungsverzeichnis ==================================
%===========================================================
\newpage
\renewcommand{\cftfigpresnum}{Abb. }
\renewcommand{\cftfigaftersnum}{:}
\setlength{\cftfignumwidth}{2cm}
\setlength{\cftfigindent}{0cm}
\listoffigures

%===========================================================
%== Tabellenverzeichnis ====================================
%===========================================================
\newpage
\renewcommand{\cfttabpresnum}{Tab. }
\renewcommand{\cfttabaftersnum}{:}
\setlength{\cfttabnumwidth}{2cm}
\setlength{\cfttabindent}{0cm}
\listoftables

%===========================================================
%== Codeverzeichnis ========================================
%===========================================================
\newpage
\renewcommand\listingscaption{Code}
\renewcommand\listoflistingscaption{Codebeispielverzeichnis}
\renewcommand{\cftfigpresnum}{Code }
\listoflistings
\addcontentsline{toc}{chapter}{Codebeispielverzeichnis}

% %===========================================================
% %== Symbolverzeichnis ======================================
% %===========================================================
% \newpage
% %% Symbolverzeichnis
% \makenomenclature
% % This will add the units
% %----------------------------------------------
% \newcommand{\nomunit}[1]{%
% \renewcommand{\nomentryend}{\hspace*{\fill}#1}}
% %----------------------------------------------
% \renewcommand{\nomname}{Symbolverzeichnis}
% %Sorgt dafür, das Symbolverzeichnis im Header steht
% \patchcmd{\thenomenclature}
%   {\chapter*{\nomname}}% usually only \chapter*{\nomname} is issued
%   {\chapter*{\nomname}\markboth{\MakeUppercase\nomname}{\MakeUppercase\nomname}}
%   {}{}
% \input{Verzeichnisse/Symbolverzeichnis.tex}
% \printnomenclature


%===========================================================
%== Abkürzungsverzeichnis ==================================
%===========================================================
%\newpage
\printglossary[type=\acronymtype, title=Abkürzungsverzeichnis, toctitle=Abkürzungsverzeichnis]



%<<<<<<<<<<<<<<<<<<<<<<<<<<<<<<<<<<<<<<<<<<<<<<<<<<<<<<<<<<<<<<<<<<<<<<<<<<<<<<<<<<<<
% Text Einleitung, Hauptteil usw. 
%<<<<<<<<<<<<<<<<<<<<<<<<<<<<<<<<<<<<<<<<<<<<<<<<<<<<<<<<<<<<<<<<<<<<<<<<<<<<<<<<<<<<

\chapter{Einleitung}

\section{Vorstellung des Kontexts der Arbeit}
Neben den etablierten Programmiersprachen, wie Java und C/C++, werden kontinuierlich neue Programmiersprachen entworfen.
Viele neue Sprachen werden als Open-Source entwickelt und auf Github gehostet.

Folgende Auflistung zeigt die auf Github beliebtesten Programmiersprachen-Projekte (Stand: Mai 2017) \cite{Github}.

% \begin{table}[H]
%     \centering
%     \begin{tabularx}{\textwidth}{|l|X|}
%     \hline 
%     \rowcolor[gray]{0.75} \textbf{Platz} & \textbf{Programmiersprache} \\
%     \hline
% 	1 & Swift \\
% 	\hline
% 	2 & Go \\
% 	\hline
% 	3 & TypeScript \\
% 	\hline  
% 	4 & Rust \\
% 	\hline
% 	5 & Kotlin \\
%     \hline
%     \end{tabularx}
%     \caption{Rangfolge der Programmiersprachen Projekte auf Github, in Anlehung an \cite{Github}}
%     \label{tab:Github}
% \end{table}

\begin{itemize}
    \item Swift
    \item Go
    \item TypeScript
    \item Rust
    \item Kotlin
\end{itemize}

Mit Go und Swift befinden sich zwei Projekte darunter, die aufgrund ihrer Herkunft besondere Aufmerksamkeit erregen.
Go wird seit 2009 von Google entwickelt und Swift wurde 2014 von Apple vorgestellt.
Doch warum sind Google und Apple daran interessiert neue Programmiersprachen zu entwickeln?
Die \gls{FAQ} von Go liefert auf die Frage ,,Why are you creating a new language?'' folgende Antwort: 

\begin{quote}
\enquote{Go was born out of frustration with existing languages and environments for systems programming.}\cite{Golang.FAQ}
\end{quote}

Rob Pike, einer der Entwickler von Go, leitet einen seiner Artikel zum Einsatz von Go bei Google, mit folgender Aussage ein:

\begin{quote}
\enquote{Go was designed to address the problems faced in software development at Google, which led to a language that is not a breakthrough research language but is nonetheless an excellent tool for engineering large software projects.}\cite{RobPike.2012}
\end{quote}

Google entwickelt Go also vor allem aus eigenem Bedarf heraus und um bestehende Probleme bei Google zu lösen.
Auf die Frage, warum Swift entwickelt wurde, gibt die offizielle Webseite von Swift folgende Antwort:

\begin{quote}
\enquote{The goal of the Swift project is to create the best available language for uses ranging from systems programming, to mobile and desktop apps, scaling up to cloud services.}\cite{Swift.Homepage}
\end{quote}

Swift verfolgt mit dem Ziel, eine der besten verfügbaren Programmiersprachen zu entwickeln, ein sehr ambitioniertes Ziel.

\section{Motivation}
Google und Apple verfolgen demnach beide das Ziel die heutigen Anforderungen an moderne Softwareentwicklung mit einer neuen Programmiersprache zu erfüllen.  
Im April 2016 war im Onlinemagazin heise.de folgende Schlagzeile zu lesen:

\begin{quote}
\enquote{Google erwägt Swift angeblich als Programmiersprache für Android}\cite{Becker}
\end{quote}

% Doch warum erwägt Google Swift als Programmiersprache für Android, wenn von Google mit Go eine eigene moderne Programmiersprache entwickelt wird?
Obwohl Google mit Go selbst eine moderne Programmiersprache entwickelt, ist Google an einem Einsatz von Swift interessiert.
Anfang 2017 war zudem auf heise.de diese Schlagzeile zu lesen:

\begin{quote}
\enquote{Programmiersprachen-Ranking: TIOBE sieht Go als Sprache des Jahres 2016}\cite{MengeSonnentag}
\end{quote}

\autoref{tab:Tiobe} zeigt die Top 20 des TIOBE-Index zum Stand Mai 2017. 
Go befindet sich auf Platz 16 und Swift belegt Platz 13. 
In der zweiten Spalte ist die Platzierung des Vorjahres zu sehen. 
Go hat sich dabei von Platz 42 auf Platz 16 verbessert.

\begin{table}[H]
    \centering
    \begin{tabularx}{\textwidth}{|c|c|X|}
    \hline 
    \rowcolor[gray]{0.75} \textbf{Mai 2017} & \textbf{Mai 2016} & \textbf{Programmiersprache} \\
    \hline
    1 & 1 & Java \\
    \hline
    2 & 2 & C \\
    \hline
    3 & 3 & C++ \\
    \hline
    4 & 5 & Python \\
    \hline
    5 & 4 & C\# \\
    \hline
    6 & 10 & Visual Basic.NET \\
    \hline
    7 & 7 & JavaScript \\
    \hline
    8 & 12 & Assembler \\
    \hline
    9 & 6 & PHP \\
    \hline
    10 & 9 & Perl \\
    \hline
    11 & 8 & Ruby \\
    \hline
    12 & 13 & Visual Basic \\
    \hline
    \rowcolor[gray]{0.9} 13 & 15 & Swift \\
    \hline
    14 & 16 & R \\
    \hline
    15 & 14 & Objective-C \\
    \hline
    \rowcolor[gray]{0.9} 16 & 42 & Go \\
    \hline
    17 & 18 & MATLAB \\	
    \hline
    18 & 11 & Delphi/Object Pascal \\	
    \hline
    19 & 19 & PL/SQL \\	
    \hline
    20 & 22 & Scratch \\	
    \hline
    \end{tabularx}
    \caption{TIOBE-Index Stand Mai 2017, in Anlehung an \cite{Tiobe}}
    \label{tab:Tiobe}
\end{table}

\section{Aufgabenstellung}
Go und Swift haben somit Interesse bei Softwareentwicklern geweckt.
In \autoref{tab:Vergleich} ist eine Gegenüberstellung von Go und Swift zu sehen.

\begin{table}[H]
    \centering
    \begin{tabularx}{\textwidth}{|X|X|}
    \hline 
    \rowcolor[gray]{0.75} \textbf{Go} & \textbf{Swift} \\
    \hline
	Google & Apple \\
	\hline
	Open Source & Open Source \\
	\hline
	Kompiliert & Kompiliert \\
	\hline  
	Statisches Typsystem & Statisches Typsystem \\
	\hline
	Garbage Collected & Garbage Collected \\
    \hline
    \end{tabularx}
    \caption{Gegenüberstellung Go und Swift}
    \label{tab:Vergleich}
\end{table}

% Der einzige Unterschied der beiden Sprachen, bei diesem Kriterien, ist, dass Go von Google und Swift von Apple entwickelt wurde. 
Ein oberflächlicher Vergleich von Go und Swift wird zu dem Ergebnis kommen, dass die beiden Sprachen gleich sind.
%Doch worin liegen die Unterschiede der beiden Programmiersprachen?
Die Unterschiede von Go und Swift müssen also im Detail liegen.
%Welche Möglichkeiten gibt es, Programmiersprachen zu vergleichen?
Ein näherer Vergleich der beiden Programmiersprachen sollte mögliche Gemeinsamkeiten und Unterschiede aufzeigen können.

Von \cite{Comparsion} wurde ein Vergleich von Programmiersprachen durchgeführt.
Einige der Vergleichskriterien dieser Arbeit sind:

\begin{itemize}
    \item Speicherverbrauch
    \item Kompilierzeit
    \item Laufzeit
    \item Größe des erzeugten Binärcodes
\end{itemize}

% Diese aufgeführten Vergleichskriterien sind zwar messbar, allerdings spiegeln sie nicht die Anforderungen an moderne Programmiersprachen wieder. 
Die aufgeführten Kriterien sind zwar messbar, allerdings sind diese Kriterien sehr abhängig von der zu Grunde liegenden Hardware.

Aus diesem Grund wurden für diese Arbeit eigene Vergleichskriterien aufgestellt, um Go und Swift miteinander zu vergleichen.
Die beiden Sprachen sollen mit diesen Kriterien miteinander verglichen werden.

\section{Aufbau der Arbeit}
Jedes Kapitel dieser Arbeit entspricht einem Vergleichskriterium.
Die zugehörigen Abschnitte konkretisieren das Vergleichskriterium.
Als nötiges Vorwissen wird vom Leser erwartet, dass er Kenntnisse in einer höheren Programmiersprache, wie Java, C\# oder C/C++, besitzt. 

\chapter{Syntax}
Die Akzeptanz einer Programmiersprache steht oftmals in Zusammenhang mit der Syntax. 
Ein Großteil der heutigen Hochsprachen ähneln in ihrer Syntax der Programmiersprache C. 
Moderne Programmiersprachen versuchen sich teilweise von diesem Muster zu lösen um zum Beispiel die Lesbarkeit des Quelltextes zu verbessern.

\section{Grundstruktur}
Um den Quelltext einer Programmiersprache in ein lauffähiges Programm zu übersetzen, muss dieser bestimmte Grundvoraussetzungen erfüllen.
% Im folgenden Abschnitt soll gezeigt werden welche Grundstruktur an Quelltext von den Compilern von Go und Swift benötigt werden. 

Bei der Einführung in eine neue Programmiersprache wird meist mit dem bekannten ,,Hallo Welt''-Beispiel begonnen.
Dieses einfache Beispiel veranschaulicht sehr gut, welche Grundstruktur eine Quelltext-Datei aufweisen muss.

Im \autoref{lst:helloWorldGo} ist ein ,,Hallo Welt''-Beispiel in Go zu sehen.
Der Quelltext beginnt mit der \emph{package}-Deklaration.
In Go ist der Quelltext in \emph{Packages} organisiert. 
Dies ist ähnlich zu \emph{Namespaces} oder \emph{Modules} in anderen Programmiersprachen.
Die Deklaration des Packages main stellt einen Spezialfall dar und deklariert eine eigenständig ausführbare Programmdatei.
Die \emph{Import}-Deklaration in Zeile 3 zeigt dem Compiler welche \emph{Packages} benötigt werden.
Das \emph{fmt}-Package wird für die Ausgabe auf der Konsole benötigt.

\begin{listing}[H]
\caption{Hallo Welt in Go}
\label{lst:helloWorldGo}
\begin{GoCode}
package main

import "fmt"

func main() {
    fmt.Println("Hallo Welt - das ist Go")
}
\end{GoCode}
\end{listing}

Go benötigt exakt die Pakete die im Quelltext verwendet werden, andernfalls kann der Quelltext nicht kompiliert werden. 
Im Gegensatz zu den meisten anderen Programmiersprachen bricht der Compiler auch ab, wenn \emph{Packages} importiert werden, die im Quelltext nicht verwendet werden.
Ähnlich zu anderen Programmiersprachen benötigt ein ausführbares Programm eine \emph{main}-Funktion.
Diese wird in Zeile 5 deklariert.
In Zeile 6 wird die Funktion \emph{Println} aus dem Package fmt aufgerufen, welche die Meldung ,,Hallo Welt - das ist Go'' auf der Konsole ausgibt \cite{Donovan.2016}.


%Swift benötigt für das gleiche Ergebnis verhältnismäßig wenige Anweisungen. 
Wie in \autoref{lst:helloWorldSwift} zu sehen, erledigt Swift die Ausgabe des Textes ,,Hallo Welt - das ist Swift'' mit nur einer Zeile Quelltext. Es müssen keine Bibliotheken eingebunden werden. Da sich die \emph{print}-Anweisung nicht in einer Funktion befindet, muss auch keine \emph{main}-Funktion definiert werden\cite{Apple.2017}.

\begin{listing}[H]
\caption{Hallo Welt in Swift}
\label{lst:helloWorldSwift}
\begin{SwiftCode}
print("Hallo Welt - das ist Swift")
\end{SwiftCode}
\end{listing}

\section{Variablen und Konstanten}
Konstanten und Variablen sind ein elementarer Bestandteil jeder Programmiersprache. 
Der Wert von Konstanten ist dem Compiler zur Compile-Zeit bekannt und kann dementsprechend evaluiert werden \cite{Donovan.2016}. 

Die Definition von Variablen und Konstanten in Go ist in \autoref{lst:VariableGo} zu sehen.
In Zeile 1 und Zeile 2 werden Variablen definiert und initialisiert.
Über das Schlüsselwort \emph{var} wird dem Compiler mitgeteilt, dass es sich um eine Variable handelt.
Eine Konstante wird mit dem Schlüsselwort \emph{const} definiert.
Im Gegensatz zu Zeile 2 ist zur Variable \emph{variable1} kein Datentyp angegben. 
Der Datentyp wird vom Compiler selbstständig erkannt. 
Dieser Mechanismus wird \emph{Type Inference} genannt und wird im \autoref{sec:TypeInference} \nameref{sec:TypeInference} näher erläutert.

\begin{listing}
\caption{Variablen und Konstanten in Go}
\label{lst:VariableGo}
\begin{GoCode}
var variable = "Variable in Go"
var variable2 string = "Noch eine Variable in Go"
const konstante = "Konstante in Go"
\end{GoCode}
\end{listing}

Im \autoref{lst:VariableSwift} ist die Definition von Variablen und Konstanten in Swift zu sehen. 
Analog zu Go verwendet auch Swift das Schlüsselwort \emph{var} zur Definition von Variablen.
Auch Swift beherrscht \emph{Type Inference} (siehe \autoref{sec:TypeInference} \nameref{sec:TypeInference}).
Eine Konstante wird in Swift mit dem Schlüsselwort \emph{let} definiert.

\begin{listing}
\caption{Variablen und Konstanten in Swift}
\label{lst:VariableSwift}
\begin{SwiftCode}
var variable1 = "Variable in Swift"
var variable2 : String = "Noch eine Variable in Swift"
let konstante = "Konstante in Swift"
\end{SwiftCode}
\end{listing}

\section{Funktionen}
Eine Funktion ermöglichte es eine Abfolge von Anweisungen zusammenzufassen und diese in einem Programm auch mehrmals aufzurufen. 
Mit Funktionen kann eine Aufgabe in kleine Teile aufgeilt werden.
Bei der Verwendung einer Funktion müssen keine Details zur Implementierung vorhanden sein.
Funktionen sind somit ein wichtiger Teil jeder Programmiersprache \cite{Kennedy.2016}

Im \autoref{lst:FunktionGo} ist der Grundaufbau einer Funktion in Go dargestellt.
Mit dem Schlüsselwort \emph{func} wird eine Funktion eingeleitet, gefolgt vom Namen der Funktion.
Eine Liste an Parametern wird in Klammern angegeben. 
Anschließend kann eine Liste von Rückgabewerte definiert werden.
Der Funktions-Körper befindet sich in geschweiften Klammern.

\begin{listing}
\caption{Aufbau einer Funktion in Go \cite{Donovan.2016}}
\label{lst:FunktionGo}
\begin{GoCode}
func name(parameter-list) (result-list){
    body
}
\end{GoCode}
\end{listing}

In \autoref{lst:BeispielFunktionGo} ist die Definition einer einfachen Funktion mit einem Rückgabewert zu sehen. 
Die Funktion \emph{add} übernimmt zwei Parameter vom Typ \emph{int} und gibt einen Wert vom Typ \emph{int} zurück.
Als Rückgabewert liefert die Funktion die Summe der beiden Parameter.
Bei nur einem Rückgabewert ist die Klammer um den Rückgabewert optional.
Sobald ein Rückgabewert definiert ist, erwartet der Compiler im Funktionskörper das \emph{return} Schlüsselwort

\begin{listing}
\caption{Beispiel-Funktionen in Go}
\label{lst:BeispielFunktionGo}
\begin{GoCode}
func add(a int, b int) (int) {
    return a + b
}
\end{GoCode}
\end{listing}

Das \autoref{lst:BeispielFunktionMultiGo} zeigt zwei Funktionen mit mehreren Rückgabewerten.
Die Funktion \emph{multiOp} verwendet Namen um die Rückgabewerte zu identifizieren.
In der Funktion können den Rückgabewerten dann Werte zugewiesen werden.
Mit dem Schlüsselwort \emph{return} werden alle Rückgabewerte in der festgelegten Reihenfolge zurückgegeben.
Im Unterschied dazu müssen in der Funktion \emph{addSubMulDiv} die gewünschten Rückgabewerte an die \emph{return}-Anweisung angehängt werden.
Hierbei muss die Reihenfolge der Rückgabewerte beachtet werden, da auch beim Aufruf der Funktion die Rückgabewerte in dieser Reihenfolge zurückgegeben werden (siehe Zeile 13). 

\begin{listing}
\caption{Beispiel-Funktionen in Go}
\label{lst:BeispielFunktionMultiGo}
\begin{GoCode}
func addSubMulDiv(a int, b int) (int, int, int, int){
    return a + b, a - b, a * b, a / b
}

func multiOp(x int, y int) (add int, sub int, mul int, div int){
    add = x + y
    sub = x - y
    mul = x * y
    div = x / y
    return
}

var a, s, m, d = addSubMulDiv(3,4)
\end{GoCode}
\end{listing}

Das \autoref{lst:FunktionenSwift} zeigt den Aufbau einer Funktion in Swift. 
Ähnlich zu Go verwendet auf Swift das Schlüsselwort \emph{func}.
Auch sonst ähnelt der Grundaufbau einer Funktion in Swift dem in Go.
Im Gegensatz zu Go muss die Definition der Rückgabewerte mit einer speziellen Zeichenkette gekennzeichnet werden.
Die Rückgabwerte werden mit dem \emph{return arrow}(->) gekennzeichnet \cite{Apple.2017}.

\begin{listing}
\caption{Funktionen in Swift}
\label{lst:FunktionenSwift}
\begin{SwiftCode}
func name(parameter-list) -> (result-list){
    body
}
\end{SwiftCode}
\end{listing}

Die Definition von Funktionen ohne Rückgabewert sind in \autoref{lst:FunktionVoidSwift} zu sehen.
Wenn keine Rückgabewert definiert werden soll, kann der \emph{return arrow}(->) weggelassen werden.
Jedoch kann der Rückgabwert auch explizit als Typ \emph{Void} definiert werden.

\begin{listing}[H]
\caption{Funktionen in Swift}
\label{lst:FunktionVoidSwift}
\begin{SwiftCode}
func ohneRueckgabe() {
    print("Hallo Welt")
}

func ohneRueckgabe() -> (Void) {
    print("Hallo Welt")
}
\end{SwiftCode}
\end{listing}

Das \autoref{lst:FunktionRueckgabewift} zeigt die Definiton einer Funktion mit einem Rückgabewert. 
Die Funktion erwartet zwei Zeichenketten und gibt eine Zeichenkette zurück.
In Zeile 5 ist der Aufruf der Funktion gezeigt. 
Swift erwartet das die Aufrufparameter mit Namen und in der definierten Reihenfolge angegeben werden.

\begin{listing}
\caption{Definiton einer Funktion mit Rückgabewert in Swift}
\label{lst:FunktionRueckgabewift}
\begin{SwiftCode}
func func1(vorName: String, nachName: String) -> (String){
    return vorName + ", " + nachName
}

print(func1(vorName:"Max", nachName:"Mustermann"))
\end{SwiftCode}
\end{listing}

Auch Swift unterstützt die Rückgabe von mehreren Rückgabewerten.
Ein Beispiel dafür ist in \autoref{lst:FunktionMultiRueckgabewift} zu sehen. 

\begin{listing}
\caption{Definiton einer Funktion mit Rückgabewert in Swift}
\label{lst:FunktionMultiRueckgabewift}
\begin{SwiftCode}
func multiOp(x:Int, y:Int) -> (add:Int, sub:Int, mul:Int, div:Int){
    let addition = x + y
    let subtraktion = x - y
    let multiplikation = x * y
    let division = x / y
	
    return (addition, subtraktion, multiplikation, division)
}

var test = multiOp(x: 3, y: 4)
print(test.add)
\end{SwiftCode}
\end{listing}

Die Funktion ist das Swift-Äquivalent zu der Funktion im \autoref{lst:BeispielFunktionMultiGo}. 
Die Rückgabeparameterr können mit einem Namen versehen werden, über den der Wert des Rückgabeparameters angesprochen werden kann. 
Die Rückgabewerte der Funktion können einer Variablen zugewiesen werden (Zeile 10).
Anschließend kann über den Namen des Rückgabeparameters auf dessen Wert zurückgegriffen werden. 
Diese Art des Zugriffs auf die einzelnen Rückgabwerte ist in Swift eleganter gelöst als in Go.





\chapter{Typsystem}
\label{ch:Typsystem}
Der Datentyp einer Variable oder Konstante beschreibt den Inhalt der Daten und teilt dem Compiler mit, wie diese Daten behandelt werden können. 
Anhand des Datentyps weiß der Compiler wieviel Arbeitsspeicher er reservieren muss und kann sicherstellen das einer Variable keine falschen Daten zugewiesen werden \cite[S.62]{Mathias.2016}.

\section{Basidatentypen}
Sowohl Go als auch Swift bringen die grundlegenden Datentypen mit, die auch von anderen Hochsprachen gewohnt sind.
In \autoref{tab:DatentypenGo} sind die in Go verfügbaren Datentypen aufgelistet.

\begin{table}[H]
    \centering
    \begin{tabularx}{\textwidth}{|l|X|X|}
    \hline 
    \rowcolor[gray]{0.75} \textbf{Datentyp} & \textbf{Beschreibung} & \textbf{Wertebereich} \\
    \hline
    bool & Wahrheitswerte & true, false \\
    \hline
    uint8 & Positive 8-Bit Ganzzahlen & 0 - 255 \\
    \hline
    uint16 & positive 16-Bit Ganzzahlen & 0 - 65535 \\
    \hline
    uint32 & positive 32-Bit Ganzzahlen	& 0 - 4294967295 \\
    \hline
    uint64 & positive 64-Bit Ganzzahlen	& 0 - 18446744073709551615 \\
    \hline
    int8 & 8-Bit Ganzzahlen & -128 - 127 \\
    \hline
    int16 & 16-Bit Ganzzahlen & -32768 - 32767 \\
    \hline
    int32 & 32-Bit Ganzzahlen & -2147483648 - 2147483647 \\
    \hline
    int64 & 64-Bit Ganzzahlen & -9223372036854775808 - 9223372036854775807 \\
    \hline
    float32 & 32-Bit Gleitpunktzahlen nach IEEE-754	& \\
    \hline
    float64 & 64-Bit Gleitpunktzahlen nach IEEE-754	& \\
    \hline
    complex64 & Komplexe Zahlen mit float32 Real- und Imaginärteil & \\
    \hline
    complex128 & Komplexe Zahlen mit float64 Real- und Imaginärteil & \\
    \hline
    byte & Alias für uint8 & 0 – 255 \\
    \hline
    rune & Alias für int32 & -2147483648 - 2147483647 \\
    \hline
    uint & positive 32- oder 64-Bit Ganzzahlen & \\
    \hline
    int & 32- oder 64-Bit Ganzzahlen & \\	
    \hline
    uintptr & Zeiger (32- oder 64-Bit) & \\	
    \hline
    \end{tabularx}
    \caption{Basisdatentypen in Go}
    \label{tab:DatentypenGo}
\end{table}

Die Ganzzahl-Datentypen können jeweils als 8-, 16-, 32- oder 64-Bit-Variante verwendet werden. 
Wird kein spezieller ganzzahliger Datentyp definiert, sondern nur ein \emph{int} oder \emph{uint} so kommt es darauf an, ob das Programm auf einem 32-Bit oder 64-Bit System kompiliert wird.
Wird auf einem 32-Bit-System eine Variable als \emph{int} deklariert ist diese jedoch nicht automatisch auch vom Typ \emph{int32}.
Im \autoref{lst:VariablenGo} ist dies beispielhaft dargestellt. 
Beim kompilieren von \autoref{lst:VariablenGo} gibt der Compiler die in Zeile 9 dargestellte Fehlermeldung aus. 
Die beiden Typen \emph{byte} und \emph{rune} sind Alias-Datentypen.
Der Typ \emph{rune} wird üblicherweise für Unicode-Zeichen benutzt, um sich semantisch von dem numerischen Datentyp \emph{int32} zu unterscheiden. 
Gleichermaßen verhält es sich mit dem Typ \emph{byte}, der dafür benutzt werden sollte Rohdaten abzulegen anstatt eines numerischen Wertes \cite[S.98]{Kennedy.2016}.

\begin{listing}[H]
\caption{Implizite und explizte Angabe des Datentypen in Go}
\label{lst:VariablenGo}
\begin{GoCode}
package main

func main() {
    var x int = 3
    var y int32 = 4

    var result = x + y	
}
//invalid operation: x + y (mismatched types int and int32)
\end{GoCode}
\end{listing}

In \autoref{tab:DatentypenSwift} ist eine Übersicht über von Swift zur Verfügung gestellten Basisdatentypen. 

\begin{table}[h]
\centering
\begin{tabularx}{\textwidth}{|l|X|X|}
 \hline 
 \rowcolor[gray]{0.75} \textbf{Datentyp} & \textbf{Beschreibung} & \textbf{Wertebereich} \\
 \hline
 Bool & Wahrheitswerte & \\ 
 \hline 
 UInt8 & Positive 8-Bit Ganzzahlen & 0 - 255\\
 \hline 
 UInt16	& positive 16-Bit Ganzzahlen & 0 - 65535\\
 \hline
 UInt32 & positive 32-Bit Ganzzahlen & 0 - 4294967295\\
 \hline
 UInt64	& positive 64-Bit Ganzzahlen & 0 - 18446744073709551615\\
 \hline
 Int8 & 8-Bit Ganzzahlen & -128 - 127 \\
 \hline
 Int16 & 16-Bit Ganzzahlen & -32768 - 32767 \\
 \hline
 Int32 & 32-Bit Ganzzahlen & -2147483648 - 2147483647 \\
 \hline
 Int64 & 64-Bit Ganzzahlen & -9223372036854775808 - 9223372036854775807 \\
 \hline
 Float & 32-Bit Gleitpunktzahlen & 6 Dezimalstellen \\
 \hline 
 Double & 64-Bit Gleitpunktzahlen & 15 Dezimalstellen \\
 \hline
\end{tabularx}
\caption{Basisdatentypen in Swift}
\label{tab:DatentypenSwift}
\end{table}

Ebenso wie Go bietet Swift dem Programmierer an eine ganzzahlige Variable mit einer bestimmten Größe von 8-, 16-, 32- oder 64-Bit zu definieren.
Das \autoref{lst:SwiftTypdelaration} ist analog zu \autoref{lst:VariablenGo} implementiert.
Auch Swift sieht in einer \emph{Int}-Variable auf einem 32-Bit-System und einer explizit als \emph{Int32} definierten Variablen einen Unterschied.
Der Swit-Compiler ist in dieser Hinsicht jedoch nicht so restriktiv wie der Go-Compiler.
Er gibt eine Warnung aus, mit dem Hinweis auf eine explizite Typumwandlung.
Das Ergebnis wird aber wie gewünscht berechnet und ausgegeben. 

\begin{listing}[H]
\caption{Implizite und explizite Datentypen in Swift}
\label{lst:SwiftTypdelaration}
\begin{SwiftCode}
var x : Int = 8
var y : Int32 = 4

print(x+y) //Ausgabe: 12
//Compiler-Meldung: 
//warning: '+' is deprecated: Mixed-type addition is deprecated. 
//Please use explicit type conversion.
\end{SwiftCode}
\end{listing}

Von \cite[S.30]{Hoffman.2017} wird empfohlen, soweit es keinen speziellen Grund gibt einen ganzahligen Typ mit einer expliziten Größe zu definieren, die impliziten Typen \emph{Int} und \emph{UInt} zu benutzen.
Diese Empfehlung kann grundsätzlich sowohl für Swift als auch Go ausgesprochen werden.
Jedoch muss der unterschiedliche Wertebereich bei 32-Bit und 64-Bit Systemen beachtet werden. 

\section{Type Inference}
\label{sec:TypeInference}
% Beide Sprachen beherrschen Type Inference. 
% Durch Type Inference ist es nicht nötig den Datentyp einer Variable anzugeben, sondern der Compiler erkennt diesen anhand des zugewiesenen Wertes. \cite[S.307]{Hinzberg.2015}
In den vorherigen Codebeispielen (\ref{lst:VariablenGo} und \ref{lst:SwiftTypdelaration}) wurden die Variablen explizit mit einem Datentyp definiert. Jedoch beherrschen sowohl Go als auch Swift einen Mechanismus der \emph{Type Inference} genannt wird. 
\emph{Type Inference} ermöglicht es, den Datentyp der Variable bei der Definition weg zu lassen. 
Stattdessen ermittelt der Compiler anhand des Initialwerts, um welchen Datentyp es sich handelt \cite[S.28]{Hoffman.2017}.
In \autoref{lst:TypeInferenceSwift} ist zu sehen, wie dies in Swift aussieht.

\begin{listing}
\caption{Type Inference in Swift}
\label{lst:TypeInferenceSwift}
\begin{SwiftCode}
var a = 1
var b : Int = 2

print(type(of: a)) //Ausgabe: Int
print(type(of: b)) //Ausgabe: Int
\end{SwiftCode}
\end{listing}

Bei der Definition der Variablen \emph{a} in Zeile 1 ist kein Datentyp angegeben, während bei der Definition der Variablen \emph{b} explizit der Datentyp \emph{Int} angegeben ist.
Die anschließende Überprüfung mit der \emph{type(of:)}-Funktion ergibt, dass beide Variablen vom Typ \emph{Int} sind.

Das \autoref{lst:TypeInferenceGo} zeigt ein Beispiel für \emph{Type Inference} in Go. 
In Zeile 9 wird eine Variabe \emph{a} ohne Angabe eines Datentyps deklariert und mit dem Wert 1 initialisiert.
Die Variable \emph{b} in Zeile 10 wird explizit mit dem Datentyp \emph{int} deklariert.
Zur Überprüfung soll der Datentyp der Variablen ausgegeben werden. 
Hierzu muss in Go das Package \emph{reflect} importiert werden. \todo{Evtl. nähere Infos zu Reflection}
Anschließend kann mit der Funktion \emph{reflect.TypeOf()} der Datentyp einer Variablen ermittelt werden (Zeile 12 und 13).
Die Ausgabe ergibt, das beide Variablen vom Typ \emph{int} sind. 
Die Variable \emph{a} wird anhand ihres Initialwerts korrekt als \emph{int} erkannt. 

\begin{listing}
\caption{Type Inference in Go \todo{Package und Import drin lassen?}}
\label{lst:TypeInferenceGo}
\begin{GoCode}
package main

import (
    "fmt"
    "reflect"
)

func main() {
    var a = 1
    var b int = 2
	
    fmt.Println(reflect.TypeOf(a)) //Ausgabe: int
    fmt.Println(reflect.TypeOf(b)) //Ausgabe: int
}
\end{GoCode}
\end{listing}

\emph{Type Inference} kann dem Programmierer oftmals die Angabe des Datentyps ersparen. 
Immer dann wenn eine Variable oder Konstante mit einem \Gls{Literal} initialisiert wird, kann der Compiler den verwendeten Datentyp eigenständig feststellen. 
Jedoch verwendet der Compiler immer den Datentyp mit dem größten Wertebereich.
In \autoref{lst:TypeInferenceSwift2} wird die Variabe \emph{a} mit dem Wert 1.1 intialisiert.
Der Compiler verwendet für die Variable den Datentyp \emph{Double}.
Die Variable \emph{b} wird explizit mit dem Datentyp \emph{Float} deklariert und mit dem gleichen Wert wie die Variable \emph{a} initialisiert.

\begin{listing}
\caption{Beispiel für Type Inference in Swift}
\label{lst:TypeInferenceSwift2}
\begin{SwiftCode}
var a = 1.1
var b : Float = 1.1

print(type(of: a)) //Ausgabe: Double
print(type(of: b)) //Ausgabe: Float
\end{SwiftCode}
\end{listing}

Der Programmierer muss bei der Entwicklung abwägen, ob er einer Variablen explizit einen Datentyp zuweist oder ob er dies dem Compiler überlässt. 
Dabei muss auf den Wertebereich einer Variablen und dem damit verbundenen Verbrauch an Arbeitsspeicher geachtet werden. 
Es sollte entschieden werden ob entweder konsequent \emph{Type Inference} verwendet wird, oder ob Variablen explizit mit einem Datentyp deklariert werden. 
Bei größeren Projekten sollte eine solche Entscheidung in den \emph{Coding Guidelines} festgehalten werden.

\section{Zeichenketten}
Zeichenketten, im Folgenden auch Strings genannt, und deren Verarbeitung sind ein wichtiger Bestandteil einer Programmiersprache. 
Quellcode-Dateien in Go müssen immer in UTF-8 codiert sein. 
Dementsprechend werden Zeichenketten in Go auch als UTF-8 interpretiert und kann Unicode-Zeichen enthalten \cite[S.117]{Donovan.2016}.

\begin{listing}
\caption{Strings in Go}
\label{lst:StringsGo}
\begin{GoCode}
var s string = "Hallo Welt"
var s2 string = `Hallo Welt`
\end{GoCode}
\end{listing}

\autoref{lst:StringsGo} zeigt die Deklaration und Initialisierung zweier Variablen mit einem String-Literal in Go.
Beide Variablen werden mit dem gleichen String-Literal initialisiert. 
Das String-Literal von Variable \emph{s} steht zwischen zwei Anführungszeichen (") während das String-Literal von Variable \emph{s2} zwischen zwei Backticks (`) steht.
Bei der Verwendung der Anführungszeichen können Escape-Sequenzen (z.B. {\textbackslash}n für einen Zeilenumbruch) verwendet werden.
Bei der Verwendung von Backticks handelt es sich um \emph{raw string literals}. 
Werden in einem \emph{raw string literal} Escape-Sequenzen verwendet, werden diese als normaler Text interpretiert \cite[S.118]{Donovan.2016}.
Das Package \emph{strings} der Standardlibrary von Go stellt eine Reihe an Funktionen bereit, um mit Strings zu arbeiten.

Die Deklaration und Initialisierung einer String-Variable in Swift ist im \autoref{lst:StringsSwift} zu sehen.
Ein String kann in Swift ebenfalls Escape-Sequenzen (z.B. {\textbackslash}n für Zeilenumbruch) enthalten.
Swift bietet derzeit keine Möglichkeit mit \emph{raw string literals} zu arbeiten. 
Die erweiterten String-Funktionen sind in Swift verfügbar, ohne eine Library zu laden.

\begin{listing}
\caption{Strings in Swift}
\label{lst:StringsSwift}
\begin{SwiftCode}
var s : String = "Hallo Welt"
\end{SwiftCode}
\end{listing}

Die Möglichkeit von Go, mit \emph{raw string literals} zu arbeiten, stellt einen Vorteil gegenüber Swift dar. 
Ein Beispiel für den Vorteil von \emph{raw string literals} wäre, das Speichern eines Ordner-Pfades in einem Windows-Dateisystem.
\autoref{lst:StringsSwift2} zeigt, wie das Speichern eines Ordner-Pfades in Swift aussieht.
Um bei der Intialisierung einen gültigen String zu erhalten, muss allen Backslashes (\textbackslash) ein weiterer Backslash vorangestellt werden. 

\begin{listing}
\caption{Strings in Swift}
\label{lst:StringsSwift2}
\begin{SwiftCode}
var s : String = "C:\\Windows\\System32\\"
\end{SwiftCode}
\end{listing}

In \autoref{lst:StringsGo2} wird die gleiche Aufgabe in Go mit einem \emph{raw string literal} gelöst. 
Die Lesbarkeit des Quelltextes wird durch den Einsatz von \emph{raw string literals} verbessert.

\begin{listing}
\caption{Strings in Go}
\label{lst:StringsGo2}
\begin{GoCode}
var s string = `C:\Windows\System32\`
\end{GoCode}
\end{listing}

\section{Arrays}

\section{Erweiterte Datenstrukturen}




\chapter{Objektorientierte Programmierung}
Das objektorientierte Programmierparadigma ist das aktuell, gerade in der Anwendungsentwicklung, am häufigsten eingesetzte Programmierparadigma.

\begin{quote}
\enquote{
Mittlerweile ist Objektorientierung so populär geworden, dass sich viele Software-Produkte, Werkzeuge und Vorgehensmodelle schon aus Marketing-Gründen objektorientiert nennen – unnötig zu sagen, dass nicht überall, wo ”objektorientiert“ draufsteht, auch ”objektorientiert“ drin ist.} 
\cite[S.16]{PoetzschHeffter.2009}
\end{quote}

Doch welche Merkmal müssen auf eine Programmiersprache zutreffen, um sie als objektorientierte Programmiersprache betiteln zu können?
Laut \cite{Lahres.2011} gelten folgende Grundelemente als Basis einer objektorientierten Programmiersprache:

\begin{itemize}
    \item Vererbung
    \item Datenkapselung
    \item Polymorphie
\end{itemize}

Erfüllen Go und Swift die Grundvoraussetzungen einer objektorientierten Programmiersprache? Dies soll in diesem Kapitel erörtert werden.

\section{Vererbung}
Ein zentrales Element der Objektorientierten Programmierung ist die Vererbung. 
Laut \cite[S. 145]{PoetzschHeffter.2009} bedeutet Vererbung im engeren Sinne, dass eine Klasse Programmteile von einer anderne Klasse automatisch übernimmt. 
Bei der einfachen Form der Vererbung, auch Einfachvererbung genannt, erbt eine Klasse von einer anderen Klasse.
Eine andere Form der Vererbung ist die sogenannte Merfachvererbung. 
Bei der Mehrfachvererbung kann eine Klasse von mehreren Klassen erben.
Von \cite[S.41]{Oestereich.1999} wird darauf hingewiesen, dass zu Vererbung auch verschieden Alternativen existieren und die Möglichkeiten und die Sinnhaftigkeit von Vererbung häufig überschätzt werden.


Dieser Abschnitt beschäftigt sich mit der Frage ob Go und Swift Vererbung unterstützen beziehungsweise ob eine Aufgabe alternativ implementiert werden kann.
Zu diesem Zweck soll eine gegebene Vererbungshierarchie in Form eines UML-Klassendiagramms beispielhaft in Go und Swift implementiert werden. 
Das folgende Beispiel orientiert sich an \cite[]{WilliamKennedy.2013}.


\section{Datenkapselung}




\section{Polymorphismus}

\chapter{Fehlerbehandlung}
Von \cite[S.310]{Apple.2017} wird Fehlerbehandlung als Prozess bezeichnet, auf Fehler zu reagieren und Ablauf des Programms im Normalzustand zu gewährleisten.
Laut \cite[S.77]{Manning.2016} ist es normal, das während der Ausführung von Computerprogrammen Fehler passieren. 
Allerdings sollte man die Möglichkeit haben diese Fehler abzufangen. 
In diesem Kapitel werden die Möglichkeiten von Go und Swift zur Fehlerbehandlung untersucht.


Um die Fehlerbehandlung in Swift zu verstehen, ist es laut \cite[S.175]{Hoffman.2017} nötig, zu verstehen, wie ein Fehler in Swift repräsentiert wird.
\autoref{lst:ErrorSwift} zeigt einen einfachen Fehler in Swift.
Der Fehler \textit{MyError} hat die drei Fehlerfälle \textit{Minor}, \textit{Bad} und \textit{Terrible}.
Der Fehler \textit{MyError} implementiert das Protocol \textit{Error}, siehe Zeile 1. 
In Zeile 4 wird bei der Definition des Fehlerfalls \textit{Terrible} mit dem Ausdruck \textit{(description: String)} angegeben, dass beim Auslösen des Fehlerfalls, zusätzliche Informationen in Form einer Fehlerbeschreibung angegeben werden können \cite[S.175]{Hoffman.2017}.

\begin{listing}
\caption{Ein einfacher Fehler in Swift \cite[S.175]{Hoffman.2017}}
\label{lst:ErrorSwift}
\begin{SwiftCode}
enum MyError : Error {
    case Minor
    case Bad 
    case Terrible (description: String)
}
\end{SwiftCode}
\end{listing}

Von \cite[S.175]{Hoffman.2017} wird argumentiert, dass die Definition eines Fehlers in Swift wesentlich einfacher und leichtgewichtiger ausfällt, als dies vergleichsweise in Java und C\# der Fall ist.
Das \textit{Exception Handling} in Sprachen wie Java wird als kostenintensive Operation angesehen, Vergleiche \cite[]{JavaExceptions}. 
Die Dokumentation von Swift äußert sich zu diesem Thema wie folgt.

\begin{quote}
\enquote{Error handling in Swift resembles exception handling in other languages, with the use of the try, catch and throw
keywords. Unlike exception handling in many languages—including Objective-C—error handling in Swift does not involve
unwinding the call stack, a process that can be computationally expensive. As such, the performance characteristics of a
throw statement are comparable to those of a return statement.} \cite[S.311]{Apple.2017}
\end{quote}

Dadurch bietet Swift die Möglichkeit ausgiebigen Gebrauch der Mechanismen zur Fehlerbehandlung zu machen, ohne dabei mit Leistungseinbußen rechnen zu müssen.
Das Schlüsselwort \textit{throw}, welches auch in Java verwendet wird, bietet die Möglichkeit einen Fehler zu ,,werfen''.
das Werfen eines Fehlers zeigt an, das etwas unerwartetes eingetreten ist und der normale Ablauf des Programms nicht fortgesetzt werden kann \cite[S.311]{Apple.2017}.
\autoref{lst:ThrowErrorSwift} zeigt das Werfen eines Fehlers mittels des Schlüsselwortes \textit{throw}.

\begin{listing}
\caption{Werfen eines Fehlers in Swift}
\label{lst:ThrowErrorSwift}
\begin{SwiftCode}
throw MyError.Terrible(description: "A Terrible Error!!!")
\end{SwiftCode}
\end{listing}

Nach dem Auslösen beziehungsweise Werfen eines Fehlers, ist es wichtig diesen Fehler zu behandeln. 
Swift kennt vier Möglichkeiten Fehler zu behandeln.

\begin{itemize}
    \item Eine Funktion wirft einen Fehler, welcher vom Aufrufer abgefangen werden kann
    \item Abfangen eines Fehlers mittels \textit{do-catch} Anweisung
    \item Mit dem \textit{try?} Schlüsselwort
    \item Mit dem \textit{try!} Schlüsselwort
\end{itemize}

In \autoref{lst:ThrowErrorFunction} ist die Definition von zwei Funktionen zu sehen. 
Soll eine Funktion in Swift einen Fehler werfen können, muss dies mit dem Schlüsselwort \textit{throws} vor Angabe des Rückgabewertes gekennzeichnet werden.
Die Funktion \textit{canThrowErrors} in Zeile 1 ist dementsprechend gekennzeichnet.
Von Apple wird eine solche Funktion als \textit{throwing Function} bezeichnet. 

\begin{listing}
\caption{Werfen eines Fehlers durch eine Funktion \cite[S311]{Apple.2017}}
\label{lst:ThrowErrorFunction}
\begin{SwiftCode}
func canThrowErrors() throws -> String

func canNotThrowErrors() -> String
\end{SwiftCode}
\end{listing}

Nur eine \textit{throwing Function} kann einen Fehler, welcher in der Funktion geworfen wird, an den Aufrufer weiter geben \cite[S.311]{Apple.2017}.
Ist eine Funktion ohne das Schlüsselwort \textit{throws} definiert, muss jeder Fehler innerhalb der Funktion abgefangen werden.
Da Swift keine Angabe von spezifischen Fehlern bei einer \textit{throwing Function} erwartet, wird von \cite[S.176]{Hoffman.2017} empfohlen alle innerhalb einer \textit{throwing Function} geworfenen Fehler zu dokumentieren, um einem Entwickler der die Funktion aufruft zu zeigen, welche Fehler er behandeln muss.
\autoref{lst:ThrowingFunctionSWift} zeigt ein Beispiel einer \textit{throwing Function} in Swift. 
Von besonderem Interesse ist hierbei die Anweisung von Zeile 6 - 8. 
Das Schlüsselwort \textit{guard} ersetzt hier den Einsatz einer \textit{if}-Fallunterscheidung.
Das gleiche Ziel könnte auch mit einer \textit{if}-Fallunterscheidung erreicht werden.
Das Schlüsselwort \textit{guard} hilft dem Entwickler dabei beispielsweise Bedingungen an Übergabeparameter von den Fallunterscheidungen in Form von Geschäftslogik \todo{Glossar} zu trennen. 

\begin{listing}
\caption{Beispiel einer \textit{throwing Function} in Swift}
\label{lst:ThrowingFunctionSWift}
\begin{SwiftCode}
enum MathErrors : Error{
    case DivisonByZero (description: String)
}

func Divide(x: Double, y: Double) throws -> Double {
    guard y != 0 else {
        throw MathErrors.DivisonByZero(description: "Divsion by Zero")
    }
	
    return x / y
}
\end{SwiftCode}
\end{listing}

Eine weitere Möglichkeit in Swift einen Fehler abzufangen ist die \textit{do-catch}-Anweisung. 
\autoref{fig:doCatchSwift} zeigt den grundsätzlichen Aufbau einer \textit{do-catch}-Anweisung.
Die \textit{do-catch}-Anweisung hat Ähnlichkeit mit der aus Java bekannten \textit{try-catch}-Anweisung.

\begin{figure}[H]
    \centering
    \includegraphics[height=8cm]{Images/doCatch.png}
    \caption{\textit{do-catch}-Anweisung in Swift \cite[S.314]{Apple.2017}}
    \label{fig:doCatchSwift}
\end{figure}

Wird ein Fehler im \textit{do}-Block geworfen, kann dieser Fehler in einem \textit{catch}-Block gefangen werden. 
Wird kein \textit{pattern} für einen \textit{catch}-Block definiert, fängt dieser \textit{catch}-Block jeden Fehler. 
\autoref{lst:doCatchSwift} zeigt die Anwendung einer \textit{do-catch}-Anweisung auf die Funktion aus \autoref{lst:ThrowingFunctionSWift}.

\begin{listing}[H]
\caption{\textit{do-catch}-Anweisung in Swift}
\label{lst:doCatchSwift}
\begin{SwiftCode}
do {
    try print(Divide(x: 6, y: 0))	
} catch MathErrors.DivisonByZero {	
    print("Error: y could not be Zero")
} catch {
    print("Error: Unknown Error")
}
\end{SwiftCode}
\end{listing}

Bei Aufruf der Funktion \textit{Divide} wird ein \textit{MathErrors.DivisonByZero}-Fehler geworfen, welcher im \textit{catch}-Block gefangen wird und die Meldung ,,Error: y could not be Zero'' ausgibt. 
Jeder andere Fehler wird im \textit{catch}-Block ohne Bedingung abgefangen und die Meldung ,,Error: Unknown Error'' ausgegeben.
Anschließend wird der Programmablauf fortgesetzt.

Laut \cite[S.181]{Hoffman.2017} gibt es bei der Arbeit mit Sprachen wie Java und C\# oftmals leere \textit{catch}-Blöcke. 
Swift beugt diesem Vorgehen mit dem \textit{try?}-Schlüsselwort vor. 
Das \autoref{lst:Try?Swift} zeigt die Anwendung des \textit{try?}-Schlüsselwortes.

\begin{listing}
\caption{Anwendng des \textit{try?}-Schlüsselwortes in Swift}
\label{lst:Try?Swift}
\begin{SwiftCode}
//Klassische Anwendung ähnlich zu Java und C#
do {
    let result = try Divide(x: 6, y: 2) 
    print("Result: \(result)")
} catch {}

//Mit Verwendung des try?-Schlüsselwortes
if let result = try? Divide(x: 6, y: 2) {
    print("Result: \(result)")
}
\end{SwiftCode}
\end{listing}

Der Konstante \textit{result} soll das Ergebnis des Funktionsaufrufs von \textit{Divide} zugewiesen werden und der Wert von \textit{result} soll anschließend ausgegben werden. 
Zeile 2 - 5 zeigt die klassische Vorgehensweise ähnlich zu Java und C\# während in Zeile 8 - 10 das \textit{try?}-Schlüsselwort verwendet wird.
Nach Ansicht von \cite[S181]{Hoffman.2017} macht die Verwendung von \textit{try?} den Quelltext übersichtlicher und einfacher lesbar.

In Java und C\# gibt es das Schlüsselwort \textit{finally} um beispielsweise nach einem \textit{try-catch}-Block abschließende Anweisungen auszuführen. 
Swift kennt hierzu das Schlüsselwort \textit{defer}.
\autoref{lst:deferSwift} zeigt die Anwendung des \textit{defer}-Schlüsselwortes.
Die Anweisung im \textit{defer}-Block (Zeile 5) wird am Ende der Ausführung des umschließenden Blocks (\textit{if-Fallunterscheidung in Zeile 2} ausgeführt, sowohl im Normalzustand als auch bei einem Fehler oder vorzeitigen Verlassen der Funktion aufgrund des Schlüsselwortes \textit{return} oder \textit{break} \cite[S.317]{Apple.2017}.
Durch das \textit{defer}-Schlüsselwort ist es möglich Aufgaben, die am Ende eines Blocks in jedem Fall auszuführen sind, an den Stellen im Quelltext zu platzieren, wo ihre Platzierung logisch ist.

\begin{listing}[H]
\caption{Anwendung von \textit{defer} in Swift Quelle: \cite[S.318]{Apple.2017}}
\label{lst:deferSwift}
\begin{SwiftCode}
func processFile(filename: String) throws {
    if exists(filename) {
        let file = open(filename)
        defer {
            close(file)
        }
        while let line = try file.readline() {
            // Work with the file.
        }
        // close(file) is called here, at the end of the scope.
    }
}
\end{SwiftCode}
\end{listing}

Im \autoref{lst:deferSwift} ist dies beispielsweise die Methode \textit{close(file)} direkt nach dem Aufruf der Methode \textit{open(filename)}. 
Dieses Verhalten ist nicht nur im Fehlerfall interessant für die Strukturierung des Quelltextes. 

Wenn davon ausgegangen werden kann, dass eine \textit{throwing Function} oder auch Methode zur Laufzeit keinen Fehler wirft, kann das Schlüsselwort \textit{try!} verwendet werden.
In \autoref{lst:Try!Swift} wird die Verwendung des \textit{try!}-Schlüsselwortes veranschaulicht.

\begin{listing}[H]
\caption{Anwendung von \textit{try!} in Swift Quelle: \cite[S.317]{Apple.2017}}
\label{lst:Try!Swift}
\begin{SwiftCode}
let photo = try! loadImage(atPath: "./Resources/John Appleseed.jpg")
\end{SwiftCode}
\end{listing}

Die Funktion von \autoref{lst:Try!Swift} wird von Apple folgendermaßen beschrieben.

\begin{quote}
\enquote{For example, the following code uses a  loadImage(atPath:) function, which loads the image
resource at a given path or throws an error if the image can’t be loaded. In this case, because the
image is shipped with the application, no error will be thrown at runtime, so it is appropriate to
disable error propagation.} \cite[S.317]{Apple.2017}
\end{quote}

Go bietet im Gegensatz zu Swift eine andere Art der Fehlerbehandlung.
In Go wird der \textit{error}-Typ benutzt um einen Fehlerzustand zu kennzeichnen \cite[]{GoBlog.ErroHandling}.
\autoref{lst:ErrorBeispielGo} zeigt ein Beispiel wie der \textit{error}-Type in Go verwendet wird.

\begin{listing}[H]
\caption{Beispiel für den \textit{error}-Typ in Go \cite[]{GoBlog.ErroHandling}}
\label{lst:ErrorBeispielGo}
\begin{GoCode}
func Open(name string) (file *File, err error)
\end{GoCode}
\end{listing}

Sollte beim Aufruf der Funktion \textit{Open} ein Fehler auftreten, gibt \textit{Open} einen Fehler in Form eines nicht leeren \textit{error}-Typs zurück.
\autoref{lst:ErrorBeispielGo2} zeigt die Anwendung des \textit{error}-Typs.

\begin{listing}[H]
\caption{Beispiel für die Anwendung des \textit{error}-Typs in Go \cite[]{GoBlog.ErroHandling}}
\label{lst:ErrorBeispielGo2}
\begin{GoCode}
f, err := os.Open("filename.ext")

if err != nil {
    log.Fatal(err)
}

// do something with the open *File f
\end{GoCode}
\end{listing}

In Zeile 3 wird überprüft ob beim Aufruf der \textit{os.Open}-Funktion ein Fehler aufgetreten ist. 
Im Fehlerfall wird die Fehlermeldung ausgegeben und der Programmablauf abgebrochen.
Der Abbruch des Programms geschieht mit dem Aufruf der Funktion \textit{log.Fatal(err)}.
Die offizielle Dokumentation von Go äußert sich wie folg zu dieser Funktion.

\begin{quote}
\enquote{Fatal is equivalent to Print() followed by a call to os.Exit(1).} \cite[]{GoDoc.Log}
\end{quote}

Auch Go besitzt die Fähigkeit eigene Fehler zu definieren.
Ein Beispiel hierfür ist in \autoref{lst:ErrorBeispielGo3} zu sehen.
Wird der Funktion \textit{Sqrt} ein negativer Wert übergeben, soll die Funktion den Wert 0 und den Fehler ,,math: square root of negative number'' zurück geben. 
Die Funktion \textit{New} aus dem Package \textit{errors} erzeugt eine neue Instanz vom Typ \textit{error}.

\begin{listing}[H]
\caption{Eigene Fehler in Go \cite[]{GoBlog.ErroHandling}}
\label{lst:ErrorBeispielGo3}
\begin{GoCode}
func Sqrt(f float64) (float64, error) {
    if f < 0 {
        return 0, errors.New("math: square root of negative number")
    }
    // implementation
}
\end{GoCode}
\end{listing}

\autoref{lst:ErrorBeispielGo4} zeigt, wie mit der Funktion \textit{errors.New} neue Fehler definiert werden können. 
Das Beispiel ist dem Quelltext von Go, im speziellen dem Package \textit{bufio} entnommen.

\begin{listing}[H]
\caption{Eigene Fehler in Go Quelle: \cite[]{GolangBufio}}
\label{lst:ErrorBeispielGo4}
\begin{GoCode}
var (
    ErrInvalidUnreadByte = errors.New("bufio: invalid use of UnreadByte")
    ErrInvalidUnreadRune = errors.New("bufio: invalid use of UnreadRune")
    ErrBufferFull        = errors.New("bufio: buffer full")
    ErrNegativeCount     = errors.New("bufio: negative count")
)
\end{GoCode}
\end{listing}

Da die in \autoref{lst:ErrorBeispielGo4} definierten Fehler können genutzt werden um spezifische Fehler zu identifizieren. 
\autoref{lst:ErrorBeispielGo5} zeigt wie die in \autoref{lst:ErrorBeispielGo4} definierten Fehler unterschieden werden können und unterschiedliche auf die Fehler reagiert werden kann.

\begin{listing}[H]
\caption{Unterscheidung von Fehlern in Go Quelle: \cite[]{GoingGo.Error1}}
\label{lst:ErrorBeispielGo5}
\begin{GoCode}
data, err := b.Peek(1)
if err != nil {
    switch err {
        case bufio.ErrNegativeCount:
            // Do something specific.
            return
        case bufio.ErrBufferFull:
            // Do something specific.
            return
        default:
            // Do something generic.
            return
    }
}
\end{GoCode}
\end{listing}

\todo{Panic}

\todo{Recover}

Die Möglichkeiten von Swift in der Fehlerbehandlung übertreffen die von Go.
Das Konzept zur Fehlerbehandlung von Swift ist dem anderer Programmiersprachen wie Java und C\# ähnlich und erweitert dies noch mit weiteren eleganten Möglichkeiten.
Das Konzept von Go hingegen ist anders und erfordert für Entwickler, die beispielsweise mit Java vertraut sind, ein gewisses Umdenken. 


\chapter{Speicherverwaltung}
Speicherverwaltung hat einen großen Einfluss darauf, wie eine Programmiersprache in der Praxis arbeitet. 
Besonders in C{\textbackslash}C++ muss der Entwickler viel Aufwand in das reservieren und freigeben von Speicher investieren.
Die heutigen Anforderung an eine nebenläufige objektorientierte Programmiersprache erfordern eine automatische Speicherverwaltung, da der Besitz eines Speicherfragments bei nebenläufigen Ausführungen nur schwer manuell zu verwalten ist \cite{RobPike.2012}.

\section{Garbage Collection}
Go verwendet zur Speicherverwaltung einen \textit{Garbage Collector}. 
Eine Meinung, die unter Entwicklern vertreten wird ist, dass \textit{Garbage Collection} Leistung kostet. 
Die FAQ von Google liefert auf die Fragestellung, warum Go mit \textit{Garbage Collection} arbeitet, folgende Antwort.

\begin{quote}
\enquote{We feel it's critical to eliminate that programmer overhead, and advances in garbage collection technology in the last few years give us confidence that we can implement it with low enough overhead and no significant latency.}\cite{Golang.FAQ}
\end{quote}

Go bietet dem Entwickler keine Möglichkeit explizit Speicher freizugeben, nur der \textit{Garbage Collector} kann Speicher freigeben \cite{RobPike.2012}.
Der \textit{Garbage Collector} von Go ist als \textit{tri-color}, \textit{mark-sweep Collector} implementiert.
Auf dem offiziellen Go Blog wir die Funktionsweise folgendermaßen beschrieben

\begin{quote}
\enquote{In a tri-color collector, every object is either white, grey, or black and we view the heap as a graph of connected objects. At the start of a GC cycle all objects are white. The GC visits all roots, which are objects directly accessible by the application such as globals and things on the stack, and colors these grey. The GC then chooses a grey object, blackens it, and then scans it for pointers to other objects. When this scan finds a pointer to a white object, it turns that object grey. This process repeats until there are no more grey objects. At this point, white objects are known to be unreachable and can be reused.} \cite{Go.GC}
\end{quote}

Swift verwendet als \textit{Garbage Collector} \textit{Automatic Reference Counting}(ARC).
\textit{Automatic Reference Counting} merkt sich die Referenzen auf Instanzen von Klassen.
Zeigt keine Referenz auf die Instanz einer Klasse, wird diese Instanz von \textit{Automatic Reference Counting} zerstört und der Speicher frei gegeben \cite[S.137]{Hoffman.2017}. 
Dieses Verhalten von \textit{Automatic Reference Counting} funktioniert in den meisten Fällen.
In seltenen Fällen muss der Entwickler seinen Quelltext anpassen um das Verhalten von \textit{Automatic Reference Counting} zu verbessern.

Wie bereits zu Anfang dieses Kapitels erwähnt, ist Speicherverwaltung besonders hinsichtlich nebenläufiger Programmierung ein wichtiger Punkt.
Go und Swift nehmen dem Programmierer durch ihre Mechanismen zur Speicherverwaltung einen Großteil der Arbeit ab. 
Ein manuelles Eingreifen in die Speicherverwaltung ist nur dann nötig, wenn im Besonderen auf Leistung geachtet werden muss oder der Programmierer gezielt die Speicherverwaltung beinflussen möchte.



\chapter{Generische Programmierung}
Generische Programmierung erlaubt es flexible und wiederverwendbare Funktionen zu schreiben. 
Sie ermöglicht es Duplikation zu vermeiden und verständlichen Quelltext zu entwickeln \cite[S.371]{Apple.2017}.


Der FAQ von Go ist zu entnehmen, das Go derzeit keine generische Programmierung unterstützt. 
Jedoch ist es möglich, das Mechanismen zur generischen Programmierung im Laufe der Zeit in Go integriert werden \cite{Golang.FAQ}.


Swift bietet direkte Untestützung für generische Programmierung.
Laut \cite[S.371]{Apple.2017} sind Generics eine der mächtigsten Funktionen von Swift und weden auch in der Standardbibliothek von Swift ausgiebig genutzt.
Programmierer welche beispielweise aus dem Java-Umfeld kommen, sollten laut \cite[S.206]{Hoffman.2017} keine Probleme haben sich mit generischer Programmierung in Swift zurecht zu finden.
\autoref{lst:GenericFunctionSwift} zeit ein Beispiel für eine generische Funktion und \autoref{lst:GenericTypeSwift} ein Beispiel für eine generische Klasse.

\begin{listing}
\caption{Generische Funktion in Swift Quelle: \cite[S.373]{Apple.2017}}
\label{lst:GenericFunctionSwift}
\begin{SwiftCode}
func swapTwoValues<T>(_ a: inout T, _ b: inout T) {
    let temporaryA = a
    a = b
    b = temporaryA
}
\end{SwiftCode}
\end{listing}

\begin{listing}
\caption{Generische Klasse in Swift Quelle: \cite[S.213]{Hoffman.2017}}
\label{lst:GenericTypeSwift}
\begin{SwiftCode}
class List<T> {
    var items = [T]()
    
    func add(item: T) {
        items.append(item)
    }
    
    func getItemAtIndex(index: Int) -> T? {
        if items.count > index {
            return items[index]
        } else {
            return nil
        }
    }
}
\end{SwiftCode}
\end{listing}


\chapter{Nebenläufigkeit}
Oftmals ist es ausreichend, das ein Programm sequenziell abläuft.
Jedoch gibt es Situationen, in denen die gleichzeitige Abarbeitung von mehreren verschiedenen Aufgaben besser ist \cite[S.128]{Kennedy.2016}.
Von \cite[S.128]{Kennedy.2016} nennt als Beispiel ein Webdienst, der mehrere Anfragen gleichzeitig behandelt.
Alle Anfragen sind einzigartig und können unabhängig voneinander behandelt werden.
Die Möglichkeit Anfragen gleichzeitig zu behandeln, kann die Leistung eines solchen Systems erheblich verbessern. 

\section{Möglichkeiten zur Nebenläufigen Programmierung}
Jede nebenläufig ausgeführte Aktion in Go wird \textit{goroutine} genannt \cite[S.352]{Donovan.2016}.
Laut \cite[S.352]{Donovan.2016} sind \textit{goroutines} ähnlinch zu \textit{Threads} \todo{Glossar?} in anderen Programmiersprachen.
Wird eine Funktion als \textit{goroutine} deklariert, wird die Funktion als unabhängige Einheit angesehen, deren Ausführung geplant wird und auf einem verfügbaren logischen Prozessor ausgeführt wird \cite[S.128]{Kennedy.2016}.

Nebenläufige Programmierung wird oft mit paralleler Programmierung gleich gesetzt. 
\cite[S.131]{Kennedy.2016} stellt klar, dass Nebenläufigkeit nichts mit Parallelität zu tun hat.
Kennedy erklärt Parallelität damit, viele Dinge auf einmal zu tun.
Nebenläufigkeit dagegen dreht sich nach Ansicht von Kennedy darum, viele Dinge auf einmal zu verwalten.
Um in Go Parallelität zu erreichen werden grundsätzlich mehrere physische Prozessoren benötigt und zusätzlich benötigt Go mehrere logische Prozessoren \cite[S.131]{Kennedy.2016}.

Ein Beispiel für die Verwendung von \textit{goroutines} wird in \autoref{lst:GoRoutine} gezeigt.

\begin{listing}[H]
\caption{Beispiel für goroutine}
\label{lst:GoRoutine}
\begin{GoCode}
import (
    "fmt"
    "sync"
)

func main() {
    var wg sync.WaitGroup
    wg.Add(1)
    
    fmt.Println("Einstig in die Main-Funktion")
    
    fmt.Println("Aufruf der goroutine")
    go func() {
        defer wg.Done()
        for i := 0; i <= 5; i++ {
            fmt.Print(i)
            fmt.Print(" ")
        }
        fmt.Println()
    }()
    
    fmt.Println("Ende der Main-Funktion")
    
    wg.Wait()
    
    fmt.Println("Programmende")
}
\end{GoCode}
\end{listing}

In Zeile 13 wird eine anonyme Funktion definiert, welche von 0 - 5 hoch zählen und die aktuelle Zahl ausgeben soll.
Durch das Schlüsselwort \textit{go} vor der Definition der Funktion wird diese Funktion zu einer \textit{goroutine}. 

\begin{listing}
\caption{Output von \autoref{lst:GoRoutine}}
\label{lst:GoRoutine2}
\begin{Commandline}
Einstig in die Main-Funktion
Aufruf der goroutine
Ende der Main-Funktion
0 1 2 3 4 5
Programmende
\end{Commandline}
\end{listing}

In \autoref{lst:GoRoutine2} ist zu sehen, dass die Ausgabe von ,,Ende der Main-Funktion'' vor der Ausgabe aus der anonymen Funktion aufgerufen wird.
Ohne die Anweisung \textit{wg.Wait()} aus Zeile 24 würde das Programm ohne die Ausgabe aus der anonymen Funktion beendet werden.
Um \textit{wg.Wait()} aufrufen zu können, werden noch die Anweisungen aus Zeile 7,8 und 14 benötigt.
Zeile 7 weist der Variable \textit{wg} ein \textit{WaitGroup} aus dem Package \textit{sync} zu.
Anschließend wird der \textit{WaitGroup} mitgeteilt, dass auf eine \textit{goroutine} gewartet werden soll. 
Die \textit{defer}-Anweisung in Zeile 14 wird am Ende der Funktion aufgerufen, auch im Fehlerfall, und teilt der \textit{WaitGroup} mit, dass die \textit{goroutine} beendet wurde.
Der Aufruf von \textit{wg.Wait()} sorgt also dafür, das auf die Beendigung der anonymen Funktion gewartet wird.

In Swift sind die Mechanismen zur nebenläufigen Programmierung unter dem Namen \textit{Grand Central Dispatch} \todo{Abkürzung?} bekannt.
GCD arbeitet mit Dispatch-Queues um abzuarbeitende Aufgaben zu verwalten.
Jede Queue verwaltet die an sie übergebenen Aufgaben und arbeitet diese der Reihenfolge nach ab (FIFO). 
Einer \textit{Dispatch-Queue} können Funktionen oder Closures übergeben werden \cite[S.254]{Hoffman.2017}.
GCD kennt drei verschieden Arten von Queues:

\begin{itemize}
    \item Serial Queues
    \item Concurrent Queues
    \item Main Dispatch Queue
\end{itemize}

Eine \textit{Serial Queue} arbeitet die Aufgaben genau in der Reihenfolge ab, in der sie an die Queue übergeben werden.
Eine \textit{Concurrent Queue} arbeitet die an sie übergebenen Aufgabe nebenläufig ab. 
Die Aufgaben in der \textit{Concurrent Queue} werden in der Reihenfolge, in der sie der Queue hinzugefügt wurden, abgearbeitet.
Die \textit{Main Dispatch Queue} ist global verfügbar und bearbeitet Aufgaben auf dem Haupt-Thread der Anwendung beziehungsweise des Programms \cite[S.255]{Hoffman.2017}.

Hoffman nennt als wichtigsten Vorteil, bei der Verwendung von \textit{Dispatch Queues}, dass das System das Erstellen und Verwalten von Threads übernimmt, und nicht wie sonst die Anwendung \cite[S.255]{Hoffman.2017}.
\autoref{lst:NebenläufigkeitSwift} zeigt ein Beispiel für nebenläufige Programmierung in Swift.

\begin{listing}[H]
\caption{Nebenläufige Programmierung in Swift in Anhlehung an \cite[S.255ff]{Hoffman.2017}}
\label{lst:NebenläufigkeitSwift}
\begin{SwiftCode}
import CoreFoundation
import Foundation
import Dispatch

func doCalc() {
    var x=100
    var y = x*x
    _ = y/x
}

func performCalculation(_ iterations: Int, tag: String) {
    let start = CFAbsoluteTimeGetCurrent()
    for _ in 0 ..< iterations {
        doCalc()
    }
    let end = CFAbsoluteTimeGetCurrent()
    print("time for \(tag): \(end-start)")
}

let concurrentQueue = DispatchQueue(label: "cqueue.hoffman.jon", 
attributes: .concurrent)

let serialQueue = DispatchQueue(label: "squeue.hoffman.jon")

let c = { performCalculation(1000, tag: "async1")}
concurrentQueue.async(execute: c)

serialQueue.async {
    performCalculation(2000, tag: "sync2")
}
\end{SwiftCode}
\end{listing}

In Zeile 16 und 19 werden jeweils eine \textit{Concurrent Queue} und eine \textit{Serial Queue} initialisiert.
Eine \textit{Dispatch Queue} wird standardmäßig als \textit{Serial Queue} angelegt, soweit nicht wie in Zeile 22 das Attribut \textit{.concurrent} angegeben wird.
Zeile 25 und 28 zeigen die zwei unterschiedlichen Möglichkeiten, wie einer \textit{Dispatch Queue} eine Aufgabe übergeben werden kann.

Go und Swift bieten dem Programmierer die Möglichkeit Nebenläufigkeit in seine Programme einzubauen. 
In Go hat Nebenläufigkeit einen hohen Stellenwert. 
Dies erkennt man beispielsweise daran, das in Go keine weiteren Bibliotheken eingebunden werden müssen, um nebenläufig zu programmieren.
Nebenläufigkeit ist direkt im Kern von Go integriert.
Swift bietet mit GCD ein einfach verständliches Modell für nebenläufige Programmierung. 

\chapter{Funktionale Programmierung}

\section{First-Class Functions}

\section{High-Oder Functions}

\section{Referenzielle Integrität}

\section{Closures}

\chapter{Anwendungsbeispiel}
Dieses Kapitel beschäftigt sich damit, die Machbarkeit der Implementierung eines konkreten Anwendungsbeispiels zu überprüfen.
Als Anwendungsbeispiel soll ein ,,Hallo Welt''-Beispiel in Form einer einfachen Webanwendung umgesetzt werden.
Zur Umsetzung soll ein Web-Framework eingesetzt werden.

% Dazu soll zuerst ein Überblick über die verfügbaren Frameworks gegeben werden und anschließend die beispielhafte Implementierung eines ,,Hallo Welt''-Beispiels in einem ausgewählten Framework für Go und Swift vorgestellt werden.

% \section{Frameworks}
% Die beiden Aufstellungen von Web-Frameworks für Go und Swift stellen lediglich einen Auszug aus den vorhandenen Frameworks dar.
% Das Augenmerk 

% Go:
% \begin{itemize}
%     \item Revel \cite{Revel} - 8326
%     \item Beego \cite{Beego} - 11105
%     \item Gin Gonic \cite{GinGonic} 10190
%     \item Iris \cite{Iris} 6946
%     \item Echo \cite{Echo} 7507
% \end{itemize}

% Swift:
% \begin{itemize}
%     \item Kitura \cite{Kitura} 5727
%     \item Perfect \cite{Perfect} 11601
%     \item Vapor \cite{Vapor} 9799
%     \item Express \cite{Express} 8534
% \end{itemize}

Zur Implementierung wurde für Go das Framework \textit{Beego}\cite{Beego} ausgewählt.
% Das Framework \textit{Beego} wurde ausgewählt, da es auf Github die meisten Follower hat.
Um mit \textit{Beego} zu arbeiten, müssen die Quelltext-Dateien des Framworks in den eigenen \textit{Go-Workspace} heruntergeladen werden.
\autoref{lst:BeegoInstallation} zeigt in Zeile 1 den Befehl zur Installation von \textit{Beego}.

\begin{listing}[H]
\caption{Installation von \textit{Beego} Quelle:\cite{Beego}}
\label{lst:BeegoInstallation}
\begin{Commandline}
go get -u github.com/astaxie/beego
\end{Commandline}
\end{listing}

\autoref{lst:HelloBeego} zeigt die die Datei \textit{hello.go}. 
Es wird ein neuer \textit{MainController} erstellt, der auf eine \textit{Get}-Anfrage ,,hello world'' ausgibt.

\begin{listing}[H]
\caption{Hello World mit Beego \\ Quelle:\cite{Beego}}
\label{lst:HelloBeego}
\begin{GoCode}
package main

import (
    "github.com/astaxie/beego"
)

type MainController struct {
    beego.Controller
}

func (this *MainController) Get() {
    this.Ctx.WriteString("hello world")
}

func main() {
    beego.Router("/", &MainController{})
    beego.Run()
}
\end{GoCode}
\end{listing}

Nachdem die Datei \textit{hello.go} mit dem Befehl ,,go build -o hello hello.go'' kompiliert wurde kann das ausführbare Programm anschließend mit dem Befehl ,,./hello'' ausgeführt werden. Daraufhin ist nach Aufruf von ,,http://localhost:8080'' im Browser der Schriftzug ,,hello world'' zu lesen.

% \begin{listing}[H]
% \caption{Kompilieren und starten Quelle:\cite{Beego}}
% \label{lst:BeegoBuild}
% \begin{Commandline}
% go build -o hello hello.go

% ./hello
% \end{Commandline}
% \end{listing}

Als Web-Framework für Swift wurde \textit{Kitura} ausgewählt.
Um \textit{Kitura} zu nutzen, muss zuerst ein neuer Projek-Workspace angelegt werden, siehe \autoref{lst:KituraInstallation}.

\begin{listing}[H]
\caption{Swift-Projekt erstellen \\ Quelle:\cite{Kitura}}
\label{lst:KituraInstallation}
\begin{Commandline}
mkdir myFirstProject
cd myFirstProject
swift package init --type executable
\end{Commandline}
\end{listing}

Die Einbindung von \textit{Kitura} erfolgt über die vom \textit{Package Manager} erstellte Datei \textit{Package.swift}, siehe \autoref{lst:KituraInstallation2} Zeile 6 - 8.

\begin{listing}[H]
\caption{\textit{Kitura} einbinden über \textit{Package.swift} \\ Quelle:\cite{Kitura}}
\label{lst:KituraInstallation2}
\begin{SwiftCode}
import PackageDescription

let package = Package(
    name: "myFirstProject",
    dependencies: [
        .Package(
            url: "https://github.com/IBM-Swift/Kitura.git", 
            majorVersion: 1, minor: 7
        )
    ])
\end{SwiftCode}
\end{listing}

Im nächsten Schritt muss die Datei \textit{main.swift}, welche auch vom \textit{Package Manager} angelegt wurde, mit dem Inhalt aus \autoref{lst:KituraMain} befüllt werden.

\begin{listing}[H]
\caption{Verwendung von \textit{Kitura} \\ Quelle:\cite{Kitura}}
\label{lst:KituraMain}
\begin{SwiftCode}
import Kitura

let router = Router() // Create a new router

// Handle HTTP GET requests to /
router.get("/") { 
    request, response, next in
    response.send("Hello, World!")
    next()
}

// Add an HTTP server and connect it to the router
Kitura.addHTTPServer(onPort: 8080, with: router)

Kitura.run() // Start the Kitura runloop (this call never returns)
\end{SwiftCode}
\end{listing}

Anschließend kann das Programm mit den Befehlen in \autoref{lst:KituraBuild} kompiliert und ausgeführt werden. 
Daraufhin kann mit dem Browser auf die Adresse ,,http://localhost:8080'' zugegriffen werden.
Nach dem Aufruf wird der Text ,,Hello, World!'' ausgegeben.

\begin{listing}[H]
\caption{Kompilieren und Starten \\ Quelle:\cite{Kitura}}
\label{lst:KituraBuild}
\begin{Commandline}
swift build

.build/debug/myFirstProject
\end{Commandline}
\end{listing}

Das einfache Anwendungsbeispiel konnte mit Go und Swift jeweils in einem geeigneten Web-Framework realisiert werden.
Der Einsatz der beiden Frameworks war einfach, bietet allerdings keine Rückschluss über die Eignung zum Einsatz in Produktivsystemen. 

\chapter{Tools}
Dieses Kapitel beschäftigt sich damit, wie eine Entwicklungsumgebung in Go und Swift aussehen könnte. 
% Welche Betriebssysteme werden unterstützt?
% Gibt es \gls{IDE} %integrierte Entwicklungsumgebungen?
% Welchen Umfang hat die Standardbibliothek?


\section{Entwicklungsumgebung}
Auf Github \cite[]{Github.Swift} werden für Swift die in \autoref{tab:UnterstützteBetriebssysteme} aufgeführten \textit{host development operationg systems} genannt.
In Go kann auf den in \autoref{tab:UnterstützteBetriebssysteme} gezeigten Plattformen entwickelt werden.

\begin{table}[H]
    \centering
    \begin{tabularx}{\textwidth}{ |X|X|X|X| }
    \hline 
    \rowcolor[gray]{0.75} \cellcolor{white} & \textbf{Windows} & \textbf{Linux} & \textbf{MacOS} \\
    \hline
    \cellcolor{Gray} \textbf{Go} Version 1.8.1 & Ab Windows XP & Ab Linux 2.6.23 & Ab MacOS X 10.8 \\
    \hline
    \cellcolor{Gray} \textbf{Swift} Version 3.1.1 &  & Ubuntu LTS und die letzte Ubuntu Version (Aktuell 16.10) & Xcode 8.3.2 \\
    \hline
    \end{tabularx}
    \caption{Unterstützte Betriebssysteme}
    \label{tab:UnterstützteBetriebssysteme}
\end{table}

Von \cite[]{TechnoPedia} wird eine Entwicklungsumgebung folgenderweise definiert.

\begin{quote}
\enquote{In software development, the development environment is a set of processes and tools that are used to develop a source code or program.}\cite[]{TechnoPedia}
\end{quote}

Eine einfache Entwicklungsumgebung besteht aus einem geeigneten Quelltext-Editor, Debugger und Compiler. Von \cite[]{NotUseIde} wird empfohlen, auf eine integrierte Entwicklungsumgebung (siehe \autoref{sec:IDE}) zu verzichten, um die hohe Lernkurve von \gls{IDE}s zu vermeiden.
Aus diesem Grund wurden alle Codebeispiele in dieser Arbeit mit einem Quelltext-Editor und den sprachspezifischen Werkzeugen entwickelt.
Als Quelltext-Editor wurde \textit{Visual Studio Code} eingesetzt. 
\textit{Visual Studio Code} zeichnet sich durch folgende Fähigkeiten aus:

\begin{itemize}
    \item Open Source
    \item Verfügbar für Windows, Linux und MacOS
    \item Durch \textit{Extensions} erweiterbar
    \item \textit{Extensions} für Go und Swift sind verfügbar
\end{itemize}

Die Installation von Swift und Go sowie Visual Studio Code wird hier nicht näher erläutert, da sich die Vorgehensweise von Version zu Version ändern kann.
Im Folgenden wird die Einrichtung einer Arbeitsumgebung, im weiteren Verlauf auch \textit{Workspace} genannt, für Go und Swift erläutert.

Die offizielle Dokumention von Go \cite[]{GoDoc.Workspaces} äußert sich folgenderweise zum allgemeinen Aufbau einer Arbeitsumgebung für Go:

\begin{itemize}
    \item \enquote{Go programmers typically keep all their Go code in a single workspace.}
    \item \enquote{A workspace contains many version control repositories (managed by Git, for example).} 
    \item \enquote{Each repository contains one or more packages.}
    \item \enquote{Each package consists of one or more Go source files in a single directory.}
    \item \enquote{The path to a package's directory determines its import path.}
\end{itemize}

Der grundsätzliche Aufbau einer Arbeitsumgebung in Go ist in \autoref{fig:GoWorkspace} zu sehen. 

\begin{figure}[H]
    \centering
    \includegraphics[height=6cm]{Images/GoWorkspace}
    \caption{Aufbau einer Arbeitsumgebung in Go}
    \label{fig:GoWorkspace}
\end{figure}

Wird Quelltext kompiliert, werden die vom Compiler erzeugten Binärdateien in den Ordner \textit{bin}, für ausführbare Progamme, und \textit{pkg}, für eigene Bibliotheken, abgelegt.

Im Ordner \textit{src} befindet sich der Quelltext, üblicherweise in Form von Repositorys eines Versionskontrollsystems.

Swift unterscheidet sich hier von Go. 
In Swift gibt es einzelne Arbeitsbereiche für jedes Projekt. 
Zudem ist es möglich, sich diese Arbeitsbereiche mit dem Tool \textit{Swift Package Manager}, welches bei der Installation von Swift mitgeliefert wird, Arbeitsbereiche automatisiert anlegen zu lassen.

\begin{listing}[H]
\caption{Anwendung des \textit{Swift Package Managers} \\ Quelle:\cite[S.22]{Hoffman.2017}}
\label{lst:SwiftPackageManager}
\begin{Commandline}
mkdir PMExample

cd PMExample

swift package init
\end{Commandline}
\end{listing}

\autoref{fig:SwiftWorkspace} zeigt den Aufbau des Arbeitsbereichs, nach Ausführen des \textit{Swift Package Managers} mit dem Befehl \textit{init}. 

\begin{figure}[H]
    \centering
    \includegraphics[height=6cm]{Images/SwiftWorkspace}
    \caption{Ein mit dem Swift Package Manager angelegter Workspace}
    \label{fig:SwiftWorkspace}
\end{figure}

Der \textit{Swift Package Manager} legt, in dem in \autoref{lst:SwiftPackageManager} angelegten Verzeichnis, die Verzeichnisse \textit{Sources} u. \textit{Tests} an.
Zusätzlich wird die Datei \textit{PMExample.swift} im Verzeichnis \textit{Sources} angelegt.

\section{Integrierte Entwicklungsumgebungen}
\label{sec:IDE}
Dieser Abschnitt soll einen Marktüberblick über die für Swift und Go vefügbaren integrierten Entwicklungsumgebungen bieten.
Eine integrierte Entwicklungsumgebung vereint laut \cite[]{TechnoPedia} die folgenden Fähigkeiten:

\begin{itemize}
    \item Editieren von Quelltext
    \item Build-Prozess ausführen
    \item Tests ausführen
    \item Debugging
\end{itemize}

Derzeit gibt es mit \textit{LiteIDE} (https://github.com/visualfc/liteide) nur eine richtige integrierte Entwicklungsumgebung für Go.
\textit{LiteIDE} ist auf folgenden Plattformen verfügbar als Open-Source verfügbar:

\begin{itemize}
    \item Windows 
    \item Linux
    \item MacOS X ab Version 10.6
    \item FreeBSD ab Version 9.2
    \item OpenBSD ab Version 5.6
\end{itemize}

Die Firma Jetbrains hat angekündigt mit \textit{Gogland}\cite[]{Gogland} eine integrierte Entwicklungsumgebung für Swift auf den Markt zu bringen, siehe \cite[]{Gogland.Heise}.
\textit{Gogland} befindet sich allerdings derzeit noch im \textit{Early Access Program}, einer Vorabversion.

Swift ist zwar für Linux und MacOS verfügbar, jedoch gibt es mit XCode nur eine integrierte Entwicklungsumgebung für MacOS.

\section{Standardbibliothek}
Apple beschreibt die Standardbibliothek von Swift mit folgendem Satz.
\begin{quote}
\enquote{The Swift language is relatively small, because many common types, functions, and operators that appear virtually everywhere in Swift code are actually defined in the Swift standard library.} \cite[S.427]{Apple.2017}
\end{quote}

Von \cite[S.184]{Kennedy.2016} wird ein Vorteil bei Verwendung der Go Standardbibliothek genannt. 

\begin{quote}
\enquote{By using packages from the standard library, you make it easier to manage your code and ensure that it’s reliable. This is because you
don’t have to worry if your program is going to break between release cycles, nor do
you have to manage third-party dependencies.} \cite[S.185]{Kennedy.2016}
\end{quote}

Die Standardbibliothek, sowohl von Go als auch Swift, ist also ein wichtiger Bestandteil der Programmiersprache. 
Laut \cite[S.185]{Kennedy.2016} beinhaltet die Standardbibliothek von Go über 100 Packages, welche in 38 Kategorien unterteilt sind.

Mit der dritten Version von Swift gelang es den Entwicklern, den Quellcode von Swift \textit{source-compatible} zu machen.
Dies bedeutet, dass der Quellcode von Swift auf allen unterstützen Plattformen (siehe \autoref{tab:UnterstützteBetriebssysteme}) kompiliert werden kann \cite[S.8]{Hoffman.2017}.
Eine ausführliche Standardbibliothek ist wichtig und minimiert die Abhängigkeit von externen Bibliotheken.





\chapter{Zusammenfassung}
\Blindtext

\chapter{Ausblick}
\Blindtext

%\chapter*{Beispiele}
\section*{Code-Test}
\blindtext

    \begin{listing}[]
    \caption{C\# Hello World}
    \label{listing:helloWorld}
    \begin{minted}[]{csharp}
    // Hello1.cs
    public class Hello1
    {
       public static void Main()
       {
          System.Console.WriteLine("Hello, World!");
       }
    }
    \end{minted}
    \end{listing}

\blindtext
\newpage

\section*{Glossar-Test}
Um einen Eintrag im Glossar zu erzeugen, muss der defninierte Glossar-Begriff z.B. \Gls{computer} aufgerufen werden

%Um eine Abkürzung zu verwenden wird \acrlong{gcd} benutzt.
Um eine Abkürzung zu verwenden wird \gls{gcd} benutzt.
\newpage


\section*{Bild-Test}
\blindtext

	\begin{figure}[h]
	\centering
	\includegraphics[width=0.5\textwidth]{Images/latex}
	\caption{Ein Beispielbild}%
	\label{figure:latex}%
	\end{figure}

Eingebunden.
\blindtext
\newpage

\section*{Tabelle-Test}
Hier kommt eine Tabelle.
Die Tabelle \ref{table:test} wird so referenziert.
\blindtext
 
\begin{table}[h]
\centering
\begin{tabular}{|c|c|c|c|} 
 \hline
 \rowcolor[gray]{0.75} \textbf{Col1} & \textbf{Col2} & \textbf{Col2} & \textbf{Col3} \\
 \hline
 1 & 6 & 87837 & 787 \\ 
 \hline 
 2 & 7 & 78 & 5415 \\
 \hline 
 3 & 545 & 778 & 7507 \\
 \hline
 4 & 545 & 18744 & 7560 \\
 \hline
 5 & 88 & 788 & 6344 \\
 \hline
\end{tabular}
\caption{Table to test captions and labels}
\label{table:test}
\end{table}

\blindtext

\newpage




%===========================================================
%== Glossar ================================================
%===========================================================
%% Glossar
\renewcommand*{\glossaryentrynumbers}[1]{} %Entfernt die Seitenzahl am Ende der Glossar-Beschreibung

\printglossary[title=Glossar,toctitle=Glossar]

%===========================================================
%== Literaturverzeichnis ===================================
%===========================================================
\printbibliography[heading=bibintoc,title={Literaturverzeichnis}]

%===========================================================
%== Anhang =================================================
%===========================================================
\appendix
\chapter{Anhang}
\newpage
\section{Ehrenwörtliche Erklärung}
\input{Anhang/EhrenwörtlicheErklärung.tex}

\end{document}
