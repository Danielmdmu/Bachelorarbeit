\chapter{Syntax}
Die Akzeptanz einer Programmiersprache steht oftmals in Zusammenhang mit der Syntax. 
Ein Großteil der heutigen Hochsprachen stammen in ihrer Syntax von der Programmiersprache C ab \cite{Wikipedia.ListProgrammingLanguages}. 
Moderne Programmiersprachen, wie Go und Swift, versuchen sich teilweise von diesem Muster zu lösen, um zum Beispiel die Lesbarkeit des Quelltextes zu verbessern.
Dies äußert sich beispielsweise darin, dass das Semikolon, welches in C dafür zuständig ist das Ende einer Anweisung zu kennzeichnen, weggelassen wird (Siehe \autoref{sec:Semikolons} \nameref{sec:Semikolons}).

\section{Grundaufbau einer Quellcodedatei}
Um den Quelltext einer Programmiersprache in ein lauffähiges Programm zu übersetzen, muss dieser bestimmte Grundvoraussetzungen erfüllen.
% Im folgenden Abschnitt soll gezeigt werden welche Grundstruktur an Quelltext von den Compilern von Go und Swift benötigt werden. 

Bei dem Erlernen einer neuen Programmiersprache wird meist mit dem bekannten ,,Hallo Welt''-Beispiel begonnen.
Dieses einfache Beispiel veranschaulicht, welche Grundstruktur eine Quellcodeatei aufweisen muss.

Im \autoref{lst:helloWorldGo} ist ein ,,Hallo Welt''-Beispiel in Go zu sehen.
Der Quelltext beginnt mit der \emph{package}-Deklaration.
In Go ist der Quelltext in \emph{Packages} organisiert. 
Dies ist ähnlich zu \emph{Namespaces} in C\# oder \emph{Packages} in Java.
Die Deklaration des Packages main stellt einen Spezialfall dar und deklariert eine eigenständig ausführbare Programmdatei.
Die \emph{Import}-Deklaration in Zeile 3 zeigt dem Compiler welche \emph{Packages} benötigt werden.
Das \emph{fmt}-Package wird für die Ausgabe auf der Konsole benötigt.

\begin{listing}[H]
\caption{Hallo Welt in Go}
\label{lst:helloWorldGo}
\begin{GoCode}
package main

import "fmt"

func main() {
    fmt.Println("Hallo Welt - das ist Go")
}
\end{GoCode}
\end{listing}

Go benötigt exakt die Pakete, die im Quelltext verwendet werden, andernfalls kann der Quelltext nicht kompiliert werden. 
Im Gegensatz zu den meisten anderen Programmiersprachen bricht der Compiler ab, wenn \emph{Packages} importiert werden, die im Quelltext nicht verwendet werden.
Ähnlich zu anderen Programmiersprachen benötigt ein ausführbares Programm eine \emph{main}-Funktion.
Diese wird in Zeile 5 deklariert.
In Zeile 6 wird die Funktion \emph{Println} aus dem Package fmt aufgerufen, welche die Meldung ,,Hallo Welt - das ist Go'' auf der Konsole ausgibt \cite{Donovan.2016}.


%Swift benötigt für das gleiche Ergebnis verhältnismäßig wenige Anweisungen. 
Wie in \autoref{lst:helloWorldSwift} zu sehen, benötigt Swift für die Ausgabe des Textes ,,Hallo Welt - das ist Swfit'' nur eine Zeile Quelltext.
Es müssen keine Bibliotheken eingebunden werden. Da sich die \emph{print}-Anweisung nicht in einer Funktion befindet, muss auch keine \emph{main}-Funktion definiert werden\cite{Apple.2017}.
Auch Swift wird kompiliert.

\begin{listing}[H]
\caption{Hallo Welt in Swift}
\label{lst:helloWorldSwift}
\begin{SwiftCode}
print("Hallo Welt - das ist Swift")
\end{SwiftCode}
\end{listing}

\section{Variablen und Konstanten}
Konstanten und Variablen sind ein elementarer Bestandteil jeder Programmiersprache. 
Der Wert von Konstanten ist dem Compiler zur Compile-Zeit bekannt und kann dementsprechend evaluiert werden \cite{Donovan.2016}. 

Die Definition von Variablen und Konstanten in Go ist in \autoref{lst:VariableGo} zu sehen.
In Zeile 1 und Zeile 2 werden Variablen definiert und initialisiert.
Über das Schlüsselwort \emph{var} wird dem Compiler mitgeteilt, dass es sich um eine Variable handelt.
Eine Konstante wird mit dem Schlüsselwort \emph{const} definiert.
Im Gegensatz zu Zeile 2 ist zur Variable \emph{variable1} kein Datentyp angegeben. 
Der Datentyp wird vom Compiler selbstständig erkannt. 
Dieser Mechanismus wird \emph{Type Inference} genannt und wird im \autoref{sec:TypeInference} \nameref{sec:TypeInference} näher erläutert.

\begin{listing}
\caption{Variablen und Konstanten in Go}
\label{lst:VariableGo}
\begin{GoCode}
var variable = "Variable in Go"
var variable2 string = "Noch eine Variable in Go"
const konstante = "Konstante in Go"
\end{GoCode}
\end{listing}

Im \autoref{lst:VariableSwift} ist die Definition von Variablen und Konstanten in Swift zu sehen. 
Analog zu Go verwendet auch Swift das Schlüsselwort \emph{var} zur Definition von Variablen.
Auch Swift beherrscht \emph{Type Inference} (siehe \autoref{sec:TypeInference} \nameref{sec:TypeInference}).
Eine Konstante wird in Swift mit dem Schlüsselwort \emph{let} definiert.

\begin{listing}
\caption{Variablen und Konstanten in Swift}
\label{lst:VariableSwift}
\begin{SwiftCode}
var variable1 = "Variable in Swift"
var variable2 : String = "Noch eine Variable in Swift"
let konstante = "Konstante in Swift"
\end{SwiftCode}
\end{listing}

\section{Funktionen}
Eine Funktion ermöglichte es eine Abfolge von Anweisungen zusammenzufassen und diese in einem Programm auch mehrmals aufzurufen. 
Mit Funktionen kann eine Aufgabe in kleine Teile aufgeteilt werden.
Bei der Verwendung einer Funktion müssen keine Details zur Implementierung vorhanden sein.
Funktionen sind somit ein wichtiger Teil jeder Programmiersprache \cite{Kennedy.2016}

Im \autoref{lst:FunktionGo} ist der Grundaufbau einer Funktion in Go dargestellt.
Mit dem Schlüsselwort \emph{func} wird eine Funktion eingeleitet, gefolgt vom Namen der Funktion.
Eine Aufzählung an Parametern wird in Klammern angegeben. 
Anschließend kann eine Liste von Rückgabewerte definiert werden.
Der Funktionskörper befindet sich in geschweiften Klammern.

\begin{listing}
\caption{Aufbau einer Funktion in Go \cite{Donovan.2016}}
\label{lst:FunktionGo}
\begin{GoCode}
func name(parameter-list) (result-list){
    body
}
\end{GoCode}
\end{listing}

In \autoref{lst:BeispielFunktionGo} ist die Definition einer einfachen Funktion mit einem Rückgabewert zu sehen. 
Die Funktion \emph{add} übernimmt zwei Parameter vom Typ \emph{int} und gibt einen Wert vom Typ \emph{int} zurück.
Als Rückgabewert liefert die Funktion die Summe der beiden Parameter.
Bei nur einem Rückgabewert ist die Klammer um den Rückgabewert optional.
Sobald ein Rückgabewert definiert ist, erwartet der Compiler im Funktionskörper das \emph{return} Schlüsselwort.

\begin{listing}
\caption{Beispiel-Funktionen in Go}
\label{lst:BeispielFunktionGo}
\begin{GoCode}
func add(a int, b int) (int) {
    return a + b
}
\end{GoCode}
\end{listing}

Das \autoref{lst:BeispielFunktionMultiGo} zeigt zwei Funktionen mit mehreren Rückgabewerten.
Die Funktion \emph{multiOp} verwendet Namen um die Rückgabewerte zu identifizieren.
In der Funktion können den Rückgabewerten dann Werte zugewiesen werden.
Mit dem Schlüsselwort \emph{return} werden alle Rückgabewerte in der festgelegten Reihenfolge zurückgegeben.
Im Unterschied dazu müssen in der Funktion \emph{addSubMulDiv} die gewünschten Rückgabewerte an die \emph{return}-Anweisung angehängt werden.
Hierbei muss die Reihenfolge der Rückgabewerte beachtet werden, da auch beim Aufruf der Funktion die Rückgabewerte in dieser Reihenfolge zurückgegeben werden (siehe Zeile 13). 

\begin{listing}
\caption{Beispiel-Funktionen in Go}
\label{lst:BeispielFunktionMultiGo}
\begin{GoCode}
func addSubMulDiv(a int, b int) (int, int, int, int){
    return a + b, a - b, a * b, a / b
}

func multiOp(x int, y int) (add int, sub int, mul int, div int){
    add = x + y
    sub = x - y
    mul = x * y
    div = x / y
    return
}

var a, s, m, d = addSubMulDiv(3,4)
\end{GoCode}
\end{listing}

Das \autoref{lst:FunktionenSwift} zeigt den Aufbau einer Funktion in Swift. 
Ähnlich zu Go verwendet auf Swift das Schlüsselwort \emph{func}.
Auch sonst ähnelt der Grundaufbau einer Funktion in Swift dem in Go.
Im Gegensatz zu Go muss die Definition der Rückgabewerte mit einer speziellen Zeichenkette gekennzeichnet werden.
Die Rückgabwerte werden mit dem \emph{return arrow}(->) gekennzeichnet \cite{Apple.2017}.

\begin{listing}
\caption{Funktionen in Swift}
\label{lst:FunktionenSwift}
\begin{SwiftCode}
func name(parameter-list) -> (result-list){
    body
}
\end{SwiftCode}
\end{listing}

Die Definition von Funktionen ohne Rückgabewert sind in \autoref{lst:FunktionVoidSwift} zu sehen.
Wenn keine Rückgabewert definiert werden soll, kann der \emph{return arrow}(->) weggelassen werden.
Jedoch kann der Rückgabwert auch explizit als Typ \emph{Void} definiert werden.

\begin{listing}[H]
\caption{Funktionen in Swift}
\label{lst:FunktionVoidSwift}
\begin{SwiftCode}
func ohneRueckgabe() {
    print("Hallo Welt")
}

func ohneRueckgabe() -> (Void) {
    print("Hallo Welt")
}
\end{SwiftCode}
\end{listing}

Das \autoref{lst:FunktionRueckgabewift} zeigt die Definiton einer Funktion mit einem Rückgabewert. 
Die Funktion erwartet zwei Zeichenketten und gibt eine Zeichenkette zurück.
In Zeile 5 ist der Aufruf der Funktion gezeigt. 
Swift erwartet das die Aufrufparameter mit Namen und in der definierten Reihenfolge angegeben werden.

\begin{listing}
\caption{Definiton einer Funktion mit Rückgabewert in Swift}
\label{lst:FunktionRueckgabewift}
\begin{SwiftCode}
func func1(vorName: String, nachName: String) -> (String){
    return vorName + ", " + nachName
}

print(func1(vorName:"Max", nachName:"Mustermann"))
\end{SwiftCode}
\end{listing}

Auch Swift unterstützt die Rückgabe von mehreren Rückgabewerten.
Ein Beispiel dafür ist in \autoref{lst:FunktionMultiRueckgabewift} zu sehen. 

\begin{listing}
\caption{Definiton einer Funktion mit Rückgabewert in Swift}
\label{lst:FunktionMultiRueckgabewift}
\begin{SwiftCode}
func multiOp(x:Int, y:Int) -> (add:Int, sub:Int, mul:Int, div:Int){
    let addition = x + y
    let subtraktion = x - y
    let multiplikation = x * y
    let division = x / y
	
    return (addition, subtraktion, multiplikation, division)
}

var test = multiOp(x: 3, y: 4)
print(test.add)
\end{SwiftCode}
\end{listing}

Die Funktion ist das Swift-Äquivalent zu der Funktion im \autoref{lst:BeispielFunktionMultiGo}. 
Die Rückgabeparameterr können mit einem Namen versehen werden, über den der Wert des Rückgabeparameters angesprochen werden kann. 
Die Rückgabewerte der Funktion können einer Variablen zugewiesen werden (Zeile 10).
Anschließend kann über den Namen des Rückgabeparameters auf dessen Wert zurückgegriffen werden. 
Diese Art des Zugriffs auf die einzelnen Rückgabwerte ist in Swift eleganter gelöst als in Go.

\section{Semikolons}
\label{sec:Semikolons}
Laut \cite{Wikipedia.ListProgrammingLanguages} gehören Go und Swift zu den Sprachen, die, was die Syntax angeht, von der Programmiersprache C abstammen.

In C muss ein Semikolon am Ende jeder Zeile stehen, um das Ende einer Anweisungen zu kennzeichnen.
In Go und Swift spielt das Semikolon eine eher untergeordnete Rolle.
In Swift sind Semikolons optional. 
Das \autoref{lst:SemicolonsSwift} zeigt zwei gültige Zeilen Quellcode in Swift.

\begin{listing}
\caption{Semikolons in Swift \cite[S.15]{Hoffman.2017}}
\label{lst:SemicolonsSwift}
\begin{SwiftCode}
print("Hello from Swift")
print("Hello from Swift");
\end{SwiftCode}
\end{listing}

\cite{Hoffman.2017} empfhielt in Swift keine Semikolons zu verwenden. 
Go benötigt ebenfalls keine Semikolons am Ende einer Quellcode-Zeile. 
Der Grund dafür wird in den offiziellen FAQ von Go erläutert. 

\begin{quote}
\enquote{
Semicolons, however, are for parsers, not for people, and we wanted to eliminate them as much as possible.
}\cite{Golang.FAQ}
\end{quote}

In Go setzt der Lexer(\todo{Glossar?}) an jedes Zeilenende, welches das Ende eines Ausdrucks sein kann, automatisch ein Semikolon. 
Dies hat zur Folge das ein gewisser Stil zur Klammerung eingehalten werden muss \cite{Golang.FAQ}.

\begin{listing}[H]
\caption{Semikolons in Go}
\label{lst:SemicolonGo}
\begin{GoCode}
package main

func add(a int, b int) (int){
    return a + b
}

func sub(a int, b int) (int)
{
    return a - b
}
//Compiler-Ausgabe: 
//main.go:8: syntax error: unexpected semicolon or newline before {
//Gofmt-Ausgabe:
//8:1: expected declaration, found '{'
\end{GoCode}
\end{listing}

In \autoref{lst:SemicolonGo} werden zwei Funktionen definiert. 
Der Go-Compiler gibt beim Kompilieren die in Zeile 12 dargestellte Fehlermeldung aus.
Bei der Funktion \emph{sub} wurde die öffnende Klammer in die nächste Zeile verschoben.
Dies ist in Go nicht möglich. 
Go bietet mit \emph{gofmt} ein Tool an, mit dem der Go-Quellcode automatisch formatiert werden kann. Bei Ausführung von \emph{gofmt} auf dem Quellcode aus \autoref{lst:SemicolonGo} wird die Fehlermeldung in Zeile 14 ausgegeben. 

