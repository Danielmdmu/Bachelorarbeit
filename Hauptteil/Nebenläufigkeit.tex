\chapter{Nebenläufigkeit}
Oftmals ist es ausreichend, das ein Programm sequenziell abläuft.
Jedoch gibt es Situationen, in denen die gleichzeitige Abarbeitung von mehreren verschiedenen Aufgaben besser ist \cite[S.128]{Kennedy.2016}.
Von \cite[S.128]{Kennedy.2016} nennt als Beispiel ein Webdienst, der mehrere Anfragen gleichzeitig behandelt.
Alle Anfragen sind einzigartig und können unabhängig voneinander behandelt werden.
Die Möglichkeit Anfragen gleichzeitig zu behandeln, kann die Leistung eines solchen Systems erheblich verbessern. 

\section{Möglichkeiten zur Nebenläufigen Programmierung}
Jede nebenläufig ausgeführte Aktion in Go wird \textit{goroutine} genannt \cite[S.352]{Donovan.2016}.
Laut \cite[S.352]{Donovan.2016} sind \textit{goroutines} ähnlinch zu \textit{Threads} \todo{Glossar?} in anderen Programmiersprachen.
Wird eine Funktion als \textit{goroutine} deklariert, wird die Funktion als unabhängige Einheit angesehen, deren Ausführung geplant wird und auf einem verfügbaren logischen Prozessor ausgeführt wird \cite[S.128]{Kennedy.2016}.

Nebenläufige Programmierung wird oft mit paralleler Programmierung gleich gesetzt. 
\cite[S.131]{Kennedy.2016} stellt klar, dass Nebenläufigkeit nichts mit Parallelität zu tun hat.
Kennedy erklärt Parallelität damit, viele Dinge auf einmal zu tun.
Nebenläufigkeit dagegen dreht sich nach Ansicht von Kennedy darum, viele Dinge auf einmal zu verwalten.
Um in Go Parallelität zu erreichen werden grundsätzlich mehrere physische Prozessoren benötigt und zusätzlich benötigt Go mehrere logische Prozessoren \cite[S.131]{Kennedy.2016}.

Ein Beispiel für die Verwendung von \textit{goroutines} wird in \autoref{lst:GoRoutine} gezeigt.

\begin{listing}[H]
\caption{Beispiel für goroutine}
\label{lst:GoRoutine}
\begin{GoCode}
import (
    "fmt"
    "sync"
)

func main() {
    var wg sync.WaitGroup
    wg.Add(1)
    
    fmt.Println("Einstig in die Main-Funktion")
    
    fmt.Println("Aufruf der goroutine")
    go func() {
        defer wg.Done()
        for i := 0; i <= 5; i++ {
            fmt.Print(i)
            fmt.Print(" ")
        }
        fmt.Println()
    }()
    
    fmt.Println("Ende der Main-Funktion")
    
    wg.Wait()
    
    fmt.Println("Programmende")
}
\end{GoCode}
\end{listing}

In Zeile 13 wird eine anonyme Funktion definiert, welche von 0 - 5 hoch zählen und die aktuelle Zahl ausgeben soll.
Durch das Schlüsselwort \textit{go} vor der Definition der Funktion wird diese Funktion zu einer \textit{goroutine}. 

\begin{listing}
\caption{Output von \autoref{lst:GoRoutine}}
\label{lst:GoRoutine2}
\begin{Commandline}
Einstig in die Main-Funktion
Aufruf der goroutine
Ende der Main-Funktion
0 1 2 3 4 5
Programmende
\end{Commandline}
\end{listing}

In \autoref{lst:GoRoutine2} ist zu sehen, dass die Ausgabe von ,,Ende der Main-Funktion'' vor der Ausgabe aus der anonymen Funktion aufgerufen wird.
Ohne die Anweisung \textit{wg.Wait()} aus Zeile 24 würde das Programm ohne die Ausgabe aus der anonymen Funktion beendet werden.
Um \textit{wg.Wait()} aufrufen zu können, werden noch die Anweisungen aus Zeile 7,8 und 14 benötigt.
Zeile 7 weist der Variable \textit{wg} ein \textit{WaitGroup} aus dem Package \textit{sync} zu.
Anschließend wird der \textit{WaitGroup} mitgeteilt, dass auf eine \textit{goroutine} gewartet werden soll. 
Die \textit{defer}-Anweisung in Zeile 14 wird am Ende der Funktion aufgerufen, auch im Fehlerfall, und teilt der \textit{WaitGroup} mit, dass die \textit{goroutine} beendet wurde.
Der Aufruf von \textit{wg.Wait()} sorgt also dafür, das auf die Beendigung der anonymen Funktion gewartet wird.

In Swift sind die Mechanismen zur nebenläufigen Programmierung unter dem Namen \textit{Grand Central Dispatch} \todo{Abkürzung?} bekannt.
GCD arbeitet mit Dispatch-Queues um abzuarbeitende Aufgaben zu verwalten.
Jede Queue verwaltet die an sie übergebenen Aufgaben und arbeitet diese der Reihenfolge nach ab (FIFO). 
Einer \textit{Dispatch-Queue} können Funktionen oder Closures übergeben werden \cite[S.254]{Hoffman.2017}.
GCD kennt drei verschieden Arten von Queues:

\begin{itemize}
    \item Serial Queues
    \item Concurrent Queues
    \item Main Dispatch Queue
\end{itemize}

Eine \textit{Serial Queue} arbeitet die Aufgaben genau in der Reihenfolge ab, in der sie an die Queue übergeben werden.
Eine \textit{Concurrent Queue} arbeitet die an sie übergebenen Aufgabe nebenläufig ab. 
Die Aufgaben in der \textit{Concurrent Queue} werden in der Reihenfolge, in der sie der Queue hinzugefügt wurden, abgearbeitet.
Die \textit{Main Dispatch Queue} ist global verfügbar und bearbeitet Aufgaben auf dem Haupt-Thread der Anwendung beziehungsweise des Programms \cite[S.255]{Hoffman.2017}.

Hoffman nennt als wichtigsten Vorteil, bei der Verwendung von \textit{Dispatch Queues}, dass das System das Erstellen und Verwalten von Threads übernimmt, und nicht wie sonst die Anwendung \cite[S.255]{Hoffman.2017}.
\autoref{lst:NebenläufigkeitSwift} zeigt ein Beispiel für nebenläufige Programmierung in Swift.

\begin{listing}[H]
\caption{Nebenläufige Programmierung in Swift in Anhlehung an \cite[S.255ff]{Hoffman.2017}}
\label{lst:NebenläufigkeitSwift}
\begin{SwiftCode}
import CoreFoundation
import Foundation
import Dispatch

func doCalc() {
    var x=100
    var y = x*x
    _ = y/x
}

func performCalculation(_ iterations: Int, tag: String) {
    let start = CFAbsoluteTimeGetCurrent()
    for _ in 0 ..< iterations {
        doCalc()
    }
    let end = CFAbsoluteTimeGetCurrent()
    print("time for \(tag): \(end-start)")
}

let concurrentQueue = DispatchQueue(label: "cqueue.hoffman.jon", 
attributes: .concurrent)

let serialQueue = DispatchQueue(label: "squeue.hoffman.jon")

let c = { performCalculation(1000, tag: "async1")}
concurrentQueue.async(execute: c)

serialQueue.async {
    performCalculation(2000, tag: "sync2")
}
\end{SwiftCode}
\end{listing}

In Zeile 16 und 19 werden jeweils eine \textit{Concurrent Queue} und eine \textit{Serial Queue} initialisiert.
Eine \textit{Dispatch Queue} wird standardmäßig als \textit{Serial Queue} angelegt, soweit nicht wie in Zeile 22 das Attribut \textit{.concurrent} angegeben wird.
Zeile 25 und 28 zeigen die zwei unterschiedlichen Möglichkeiten, wie einer \textit{Dispatch Queue} eine Aufgabe übergeben werden kann.

Go und Swift bieten dem Programmierer die Möglichkeit Nebenläufigkeit in seine Programme einzubauen. 
In Go hat Nebenläufigkeit einen hohen Stellenwert. 
Dies erkennt man beispielsweise daran, das in Go keine weiteren Bibliotheken eingebunden werden müssen, um nebenläufig zu programmieren.
Nebenläufigkeit ist direkt im Kern von Go integriert.
Swift bietet mit GCD ein einfach verständliches Modell für nebenläufige Programmierung. 