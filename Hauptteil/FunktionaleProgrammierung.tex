\chapter{Funktionale Programmierung}
Um zu analysieren, inwieweit Swift und Go funktionale Programmierung ermögliche, muss zunächst erläutert werden, was unter funktionaler Programmierung zu verstehen ist.
Piepmeyer beantwortet in \cite[S.6]{Piepmeyer.2010} die Frage ,,Wann ist eine Sprache Funktional'' folgendermaßen:

\begin{itemize}
    \item In funktionalen Sprachen können Funktionen anonym definiert werden
    \item Funktionen werden wie alle anderen Daten behandelt
\end{itemize}

Von Esser werden in \cite[S.243]{Esser.2011} zwei Begriffe genannt, welche funktionale Programmiersprachen erfüllen und bei prozeduralen Programmiersprachen fehlen.

\begin{itemize}
    \item First-Class Function
    \item High-Order Function
\end{itemize}

Von Piepmeyer wird ein erheblicher Vorteil der funktionalen Programmierung genannt.

\begin{quote}
\enquote{In der funktionalen Programmierung hängt das Ergebnis einer Berechnung nicht vom Zeitpunkt der Berechnung ab. Aufgrund dieser referentiellen Transparenz können funktionale Programme leicht parallelisiert werden.} \cite[S.13]{Piepmeyer.2010}
\end{quote}

% Dieses Kapitel soll zeigen, ob mit Go und Swift funktional programmiert werden kann.
Nachfolgend soll erläutert werden, ob mit Go und Swift funktional programmiert werden kann.

\section{Anonyme Funktionen}
Ein Merkmal von funktionalen Programmiersprachen ist die Möglichkeit eine Funktion anonym, also ohne Namen, zu definieren \cite[S.28]{Piepmeyer.2010}.
% Anonyme Funktionen sind auch als \textit{Closures} beziehungsweise \textit{Lambda Expressions} bekannt.
\cite[S.219]{Hoffman.2017} sieht anonyme Funktionen als Datentyp, der anstatt beispielsweise ganzzahligen Werten, einen Block Quellcode beinhaltet.

\begin{listing}[H]
\caption{Anonyme Funktion in Swift \\ Quelle:\cite[S.220]{Hoffman.2017}}
\label{lst:SwiftClosure}
\begin{SwiftCode}
let clos1 = {
    () -> Void in
    print("Hello World")
}
\end{SwiftCode}
\end{listing}

Ein Beispiel für eine anonyme Funktion in Go ist in \autoref{lst:GoFirstClassFunctions} in Zeile 15 - 17 zu sehen.

\section{First-Class Functions}
Um ein Gefühl für den Begriff \textit{First-Class Function} zu bekommen, leitet \cite[S.243f]{Esser.2011} das Thema mit einer Definition des Begriff \textit{First-Class Objects} aus der Objektorientierung ein.
Objekte werden als Werte gesehen, die über Referenzen, Parameter oder als Ergebnis weitergereicht werden.
Die Methoden eines Objekts sind nicht eigenständig, sondern können nur indirekt über das Objekt weitergegeben werden.
Der Begriff \textit{First-Class Function} dreht diese Sichtweise um.
Funktionen sind Werte, die über Referenzen, Parameter oder als Ergebnis an andere Funktionen weitergegeben werden.
Funktionen sind eigenständig und nicht an Objekte gebunden. 

\autoref{lst:GoFirstClassFunctions} zeigt die Anwendung von \textit{First-Class Functions} in Go.

\begin{listing}[H]
\caption{\textit{First-Class Functions} in Go \\ Quelle:\cite[]{Go.FirstClassFunctions}}
\label{lst:GoFirstClassFunctions}
\begin{GoCode}
package main

import "fmt"

func CallWith(f func(string), who string) {
    f(who)
}

type FunctionHolder struct {
    Function func(string)
}

func main() {
    holder :=   FunctionHolder{ 
                    func(who string) {
                        fmt.Println("Hello,", who) 
                    }
                }
                
    CallWith(holder.Function,"ernest")
}
\end{GoCode}
\end{listing}

In Zeile 5 - 7 wird eine Funktion \textit{CallWith} definiert, welche als Parameter eine Funktion und einen String übernimmt und anschließend die übergebene Funktion mit dem übergebenen String aufruft.
In Zeile 9 - 1 wird eine neuer Typ \textit{FunctionHolder} definiert, welcher eine Funktion mit einem String als Parameter aufnimmt.
In der \textit{main}-Funktion (Ab Zeile 13) wird zuerst eine neue Instanz des Typs \textit{FunctionHolder} erstellt und dieser mit einer anonymen Funktion, welche einen String entgegen nimmt, initialisiert.
Anschließend wird die Funktion \textit{CallWith} mit der in \textit{holder} gespeicherten Funktion und dem Literal ,,ernest'' aufgerufen.
Als Ergebnis wird ,,Hello, ernest'' ausgegeben. 

\section{High-Order Functions}
\textit{High-Order Functions} sind Funktionen, die eine Funktion als Übergabeparameter haben oder eine Funktion als Ergebnis liefern \cite[S.63]{Piepmeyer.2010}.

Im \autoref{lst:GoHighOrderFunctions} ist ein Beispiel für \textit{High-Order Functions} in Go zu sehen. 
In Zeile 11 wird ein neuer Type Namens \textit{strategy} definiert. 
Der Datentyp von \textit{strategy} ist eine Funktion.
Der Rückgabewert von \textit{strategy} ist vom Typ \textit{action} (Zeile 8), welcher wiederum eine Funktion darstellt.

\begin{listing}[H]
\caption{\textit{High-Order Functions} in Go \\ Quelle:\cite[]{Go.HighOrderFunctions}}
\label{lst:GoHighOrderFunctions}
\begin{GoCode}
// A score includes scores accumulated in previous turns for each player,
// as well as the points scored by the current player in this turn.
type score struct {
    player, opponent, thisTurn int
}

// An action transitions stochastically to a resulting score.
type action func(current score) (result score, turnIsOver bool)

// A strategy chooses an action for any given score.
type strategy func(score) action
\end{GoCode}
\end{listing}

\autoref{lst:SwiftHighOrderFunctions} zeigt ein Beispiel für \textit{High-Order Functions} in Swift.
Die Funktion \textit{aHigherOrderFunction} übernimmt als Parameter eine anonyme Funktion mit dem Name \textit{closure}.
In Zeile 5 - 7 wird eine anonyme Funktion definiert und der Konstante \textit{myClosure} zugewiesen.
In Zeile 9 wird \textit{myClosure} der Funktion \textit{aHigherOrderFunction} übergeben.

\begin{listing}[H]
\caption{\textit{High-Order Functions} in Swift \\ Quelle:\cite[]{Swift.HighOrderFunctions}}
\label{lst:SwiftHighOrderFunctions}
\begin{SwiftCode}
func aHigherOrderFunction(closure: () -> Void) {
    closure()
}

let myClosure = {
    print("Hello!")
}

aHigherOrderFunction(myClosure) // prints: Hello
\end{SwiftCode}
\end{listing}



% \section{Referentielle Transparenz}
% Als \textit{referentielle Transparenz} bezeichnet man die Eigenschaft einer Funktion, nur von den Werten seiner Teilfunktionen abhängig zu sein und nicht vom Zeitpunkt der Berechnung \cite[S.11]{Piepmeyer.2010}.


