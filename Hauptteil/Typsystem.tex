\chapter{Typsystem}
\label{ch:Typsystem}
Der Datentyp einer Variablen oder Konstanten beschreibt den Inhalt der Daten und teilt dem Compiler mit, wie diese Daten behandelt werden können. 
Anhand des Datentyps weiß der Compiler, wieviel Arbeitsspeicher er reservieren muss und kann sicherstellen, dass einer Variablen keine falschen Daten zugewiesen werden \cite[S.62]{Mathias.2016}.

\section{Basisdatentypen}
Sowohl Go als auch Swift beinhalten die grundlegenden Datentypen, die auch von anderen Hochsprachen bekannt sind.
In \autoref{tab:DatentypenGo} sind die in Go verfügbaren Datentypen aufgelistet.

\begin{table}[H]
    \centering
    \begin{tabularx}{\textwidth}{|l|X|X|}
    \hline 
    \rowcolor[gray]{0.75} \textbf{Datentyp} & \textbf{Beschreibung} & \textbf{Wertebereich} \\
    \hline
    bool & Wahrheitswerte & true, false \\
    \hline
    uint8 & Positive 8-Bit Ganzzahlen & 0 - 255 \\
    \hline
    uint16 & positive 16-Bit Ganzzahlen & 0 - 65535 \\
    \hline
    uint32 & positive 32-Bit Ganzzahlen	& 0 - 4294967295 \\
    \hline
    uint64 & positive 64-Bit Ganzzahlen	& 0 - 18446744073709551615 \\
    \hline
    int8 & 8-Bit Ganzzahlen & -128 - 127 \\
    \hline
    int16 & 16-Bit Ganzzahlen & -32768 - 32767 \\
    \hline
    int32 & 32-Bit Ganzzahlen & -2147483648 - 2147483647 \\
    \hline
    int64 & 64-Bit Ganzzahlen & -9223372036854775808 - 9223372036854775807 \\
    \hline
    float32 & 32-Bit Gleitpunktzahlen nach IEEE-754	& \\
    \hline
    float64 & 64-Bit Gleitpunktzahlen nach IEEE-754	& \\
    \hline
    complex64 & Komplexe Zahlen mit float32 Real- und Imaginärteil & \\
    \hline
    complex128 & Komplexe Zahlen mit float64 Real- und Imaginärteil & \\
    \hline
    byte & Alias für uint8 & 0 – 255 \\
    \hline
    rune & Alias für int32 & -2147483648 - 2147483647 \\
    \hline
    uint & Positive 32- oder 64-Bit Ganzzahlen & \\
    \hline
    int & 32- oder 64-Bit Ganzzahlen & \\	
    \hline
    uintptr & Zeiger (32- oder 64-Bit) & \\	
    \hline
    \end{tabularx}
    \caption{Basisdatentypen in Go}
    \label{tab:DatentypenGo}
\end{table}

Die Ganzzahl-Datentypen können jeweils als 8-, 16-, 32- oder 64-Bit-Variante verwendet werden. 
Wird kein spezieller, ganzzahliger Datentyp definiert, sondern ein \emph{int} oder \emph{uint} so kommt es darauf an, ob das Programm auf einem 32-Bit oder 64-Bit System kompiliert wird.
Wird auf einem 32-Bit-System eine Variable als \emph{int} deklariert, ist diese  nicht automatisch auch vom Typ \emph{int32}.
Im \autoref{lst:VariablenGo} ist dies beispielhaft dargestellt. 
Beim Kompilieren von \autoref{lst:VariablenGo} gibt der Compiler die in Zeile 9 dargestellte Fehlermeldung aus. 
Die beiden Typen \emph{byte} und \emph{rune} sind Alias-Datentypen.
Der Typ \emph{rune} wird üblicherweise für Unicode-Zeichen benutzt, um sich semantisch von dem numerischen Datentyp \emph{int32} zu unterscheiden. 
Gleichermaßen verhält es sich mit dem Typ \emph{byte}, der dafür benutzt werden sollte, unverarbeitete Daten abzulegen, anstatt eines numerischen Wertes \cite[S.98]{Kennedy.2016}.

\begin{listing}[H]
\caption{Implizite und explizite Datentypen in Go}
\label{lst:VariablenGo}
\begin{GoCode}
package main

func main() {
    var x int = 3
    var y int32 = 4

    var result = x + y	
}
//invalid operation: x + y (mismatched types int and int32)
\end{GoCode}
\end{listing}

In \autoref{tab:DatentypenSwift} ist eine Übersicht über die von Swift zur Verfügung gestellten Basisdatentypen. 

\begin{table}[H]
\centering
\begin{tabularx}{\textwidth}{|l|X|X|}
 \hline 
 \rowcolor[gray]{0.75} \textbf{Datentyp} & \textbf{Beschreibung} & \textbf{Wertebereich} \\
 \hline
 Bool & Wahrheitswerte & \\ 
 \hline 
 UInt8 & Positive 8-Bit Ganzzahlen & 0 - 255\\
 \hline 
 UInt16	& positive 16-Bit Ganzzahlen & 0 - 65535\\
 \hline
 UInt32 & positive 32-Bit Ganzzahlen & 0 - 4294967295\\
 \hline
 UInt64	& positive 64-Bit Ganzzahlen & 0 - 18446744073709551615\\
 \hline
 Int8 & 8-Bit Ganzzahlen & -128 - 127 \\
 \hline
 Int16 & 16-Bit Ganzzahlen & -32768 - 32767 \\
 \hline
 Int32 & 32-Bit Ganzzahlen & -2147483648 - 2147483647 \\
 \hline
 Int64 & 64-Bit Ganzzahlen & -9223372036854775808 - 9223372036854775807 \\
 \hline
 Float & 32-Bit Gleitpunktzahlen & 6 Dezimalstellen \\
 \hline 
 Double & 64-Bit Gleitpunktzahlen & 15 Dezimalstellen \\
 \hline
\end{tabularx}
\caption{Basisdatentypen in Swift}
\label{tab:DatentypenSwift}
\end{table}

Ebenso wie Go, bietet Swift dem Programmierer an, eine ganzzahlige Variable mit einer bestimmten Größe von 8-, 16-, 32- oder 64-Bit zu definieren.
Das \autoref{lst:SwiftTypdelaration} ist analog zu \autoref{lst:VariablenGo} implementiert.
Auch Swift sieht in einer \emph{Int}-Variable auf einem 32-Bit-System und einer explizit als \emph{Int32} definierten Variablen einen Unterschied.
Der Swift-Compiler ist in dieser Hinsicht jedoch nicht so restriktiv wie der Go-Compiler.
Er gibt eine Warnung mit dem Hinweis auf eine explizite Typumwandlung aus.
Das Ergebnis wird aber wie gewünscht berechnet und ausgegeben. 

\begin{listing}[H]
\caption{Implizite und explizite Datentypen in Swift}
\label{lst:SwiftTypdelaration}
\begin{SwiftCode}
var x : Int = 8
var y : Int32 = 4

print(x+y) //Ausgabe: 12
//Compiler-Meldung: 
//warning: '+' is deprecated: Mixed-type addition is deprecated. 
//Please use explicit type conversion.
\end{SwiftCode}
\end{listing}

Von \cite[S.30]{Hoffman.2017} wird empfohlen, soweit es keinen speziellen Grund gibt, einen ganzzahligen Typ mit einer expliziten Größe zu definieren, die impliziten Typen \emph{Int} und \emph{UInt} zu benutzen.
Diese Empfehlung kann grundsätzlich sowohl für Swift als auch Go ausgesprochen werden.
Jedoch muss der unterschiedliche Wertebereich bei 32-Bit und 64-Bit Systemen beachtet werden. 

\section{Type Inference}
\label{sec:TypeInference}
% Beide Sprachen beherrschen Type Inference. 
% Durch Type Inference ist es nicht nötig den Datentyp einer Variable anzugeben, sondern der Compiler erkennt diesen anhand des zugewiesenen Wertes. \cite[S.307]{Hinzberg.2015}
In den vorherigen Codebeispielen (\ref{lst:VariablenGo} und \ref{lst:SwiftTypdelaration}) werden die Variablen explizit mit einem Datentyp definiert. Jedoch beherrschen sowohl Go als auch Swift einen Mechanismus, der \emph{Type Inference} genannt wird. 
\emph{Type Inference} ermöglicht es, den Datentypen der Variablen bei der Definition weg zu lassen. 
Stattdessen ermittelt der Compiler anhand des Initialwerts, um welchen Datentyp es sich handelt \cite[S.28]{Hoffman.2017}.
In \autoref{lst:TypeInferenceSwift} ist zu sehen, wie dies in Swift aussieht.

\begin{listing}[H]
\caption{Type Inference in Swift}
\label{lst:TypeInferenceSwift}
\begin{SwiftCode}
var a = 1
var b : Int = 2

print(type(of: a)) //Ausgabe: Int
print(type(of: b)) //Ausgabe: Int
\end{SwiftCode}
\end{listing}

Bei der Definition der Variablen \emph{a} in Zeile 1 ist kein Datentyp angegeben, während bei der Definition der Variablen \emph{b} explizit der Datentyp \emph{Int} angegeben ist.
Die anschließende Überprüfung mit der \emph{type(of:)}-Funktion ergibt, dass beide Variablen vom Typ \emph{Int} sind.

Das \autoref{lst:TypeInferenceGo} zeigt ein Beispiel für \emph{Type Inference} in Go. 
In Zeile 9 wird eine Variable \emph{a} ohne Angabe eines Datentyps deklariert und mit dem Wert 1 initialisiert.
Die Variable \emph{b} in Zeile 10 wird explizit mit dem Datentyp \emph{int} deklariert.
Zur Überprüfung soll der Datentyp der Variablen ausgegeben werden. 
Hierzu muss in Go das Package \emph{reflect} importiert werden. 
Anschließend kann mit der Funktion \emph{reflect.TypeOf()} der Datentyp einer Variablen ermittelt werden (Zeile 12 und 13).
Die Ausgabe ergibt, das beide Variablen vom Typ \emph{int} sind. 
Die Variable \emph{a} wird anhand ihres Initialwerts korrekt als \emph{int} erkannt. 

\begin{listing}[H]
\caption{Type Inference in Go}
\label{lst:TypeInferenceGo}
\begin{GoCode}
package main

import (
    "fmt"
    "reflect"
)

func main() {
    var a = 1
    var b int = 2
	
    fmt.Println(reflect.TypeOf(a)) //Ausgabe: int
    fmt.Println(reflect.TypeOf(b)) //Ausgabe: int
}
\end{GoCode}
\end{listing}

\emph{Type Inference} kann dem Programmierer oftmals die Angabe des Datentyps ersparen. 
Immer dann, wenn eine Variable oder Konstante mit einem Literal initialisiert wird, kann der Compiler den verwendeten Datentyp eigenständig feststellen. 
Jedoch verwendet der Compiler immer den Datentyp mit dem größten Wertebereich.
In \autoref{lst:TypeInferenceSwift2} wird die Variabe \emph{a} mit dem Wert 1.1 intialisiert.
Der Compiler verwendet für die Variable den Datentyp \emph{Double}.
Die Variable \emph{b} wird explizit mit dem Datentyp \emph{Float} deklariert und mit dem gleichen Wert wie die Variable \emph{a} initialisiert.

\begin{listing}[H]
\caption{Beispiel für Type Inference in Swift}
\label{lst:TypeInferenceSwift2}
\begin{SwiftCode}
var a = 1.1
var b : Float = 1.1

print(type(of: a)) //Ausgabe: Double
print(type(of: b)) //Ausgabe: Float
\end{SwiftCode}
\end{listing}

Der Programmierer muss bei der Entwicklung abwägen, ob er einer Variablen explizit einen Datentyp zuweist, oder ob er dies dem Compiler überlässt. 
Dabei muss auf den Wertebereich einer Variablen und dem damit verbundenen Verbrauch an Arbeitsspeicher geachtet werden. 
Es sollte entschieden werden, ob entweder konsequent \emph{Type Inference} verwendet wird, oder ob Variablen explizit mit einem Datentyp deklariert werden. 
Bei größeren Projekten sollte eine solche Entscheidung als Richtlinie festgehalten werden.

\section{Zeichenketten}
Zeichenketten, im Folgenden auch Strings genannt, und deren Verarbeitung sind ein wichtiger Bestandteil einer Programmiersprache. 
Quellcode-Dateien in Go müssen immer in UTF-8 codiert sein. 
Dementsprechend werden Zeichenketten in Go auch als UTF-8 interpretiert und können Unicode-Zeichen enthalten \cite[S.117]{Donovan.2016}.

\begin{listing}[H]
\caption{Strings in Go}
\label{lst:StringsGo}
\begin{GoCode}
var s string = "Hallo Welt"
var s2 string = `Hallo Welt`
\end{GoCode}
\end{listing}

\autoref{lst:StringsGo} zeigt die Deklaration und Initialisierung zweier Variablen mit einem String-Literal in Go.
Beide Variablen werden mit dem gleichen String-Literal initialisiert. 
Das String-Literal von Variable \emph{s} steht zwischen zwei Anführungszeichen (") während das String-Literal von Variable \emph{s2} zwischen zwei Backticks (`) steht.
Bei der Verwendung der Anführungszeichen können Escape-Sequenzen (z.B. {\textbackslash}n für einen Zeilenumbruch) verwendet werden.
Bei der Verwendung von Backticks handelt es sich um \emph{raw string literals}. 
Werden in einem \emph{raw string literal} Escape-Sequenzen verwendet, werden diese als normaler Text interpretiert \cite[S.118]{Donovan.2016}.
Das Package \emph{strings} der Standardlibrary von Go stellt eine Reihe an Funktionen bereit, um mit Strings zu arbeiten.

Die Deklaration und Initialisierung einer String-Variable in Swift ist im \autoref{lst:StringsSwift} zu sehen.
Ein String kann in Swift ebenfalls Escape-Sequenzen (z.B. {\textbackslash}n für Zeilenumbruch) enthalten.
Swift bietet derzeit keine Möglichkeit mit \emph{raw string literals} zu arbeiten. 
Die erweiterten String-Funktionen sind in Swift verfügbar, ohne eine Library zu laden.

\begin{listing}[H]
\caption{Strings in Swift I}
\label{lst:StringsSwift}
\begin{SwiftCode}
var s : String = "Hallo Welt"
\end{SwiftCode}
\end{listing}

Die Möglichkeit von Go, mit \emph{raw string literals} zu arbeiten, stellt einen Vorteil gegenüber Swift dar. 
Ein Beispiel für den Vorteil von \emph{raw string literals} wäre das Speichern eines Ordner-Pfades in einem Windows-Dateisystem.
\autoref{lst:StringsSwift2} zeigt, wie das Speichern eines Ordner-Pfades in Swift aussieht.
Um bei der Intialisierung einen gültigen String zu erhalten, muss allen Backslashes (\textbackslash) ein weiterer Backslash vorangestellt werden. 

\begin{listing}[H]
\caption{Strings in Swift II}
\label{lst:StringsSwift2}
\begin{SwiftCode}
var s : String = "C:\\Windows\\System32\\"
\end{SwiftCode}
\end{listing}

In \autoref{lst:StringsGo2} wird die gleiche Aufgabe in Go mit einem \emph{raw string literal} gelöst. 
Die Lesbarkeit des Quelltextes wird durch den Einsatz von \emph{raw string literals} verbessert.

\begin{listing}[H]
\caption{\textit{raw string literal} in Go}
\label{lst:StringsGo2}
\begin{GoCode}
var s string = `C:\Windows\System32\`
\end{GoCode}
\end{listing}



