\chapter{Objektorientierte Programmierung}
Das objektorientierte Programmierparadigma ist das aktuell, gerade in der Anwendungsentwicklung, am häufigsten eingesetzte Programmierparadigma.

\begin{quote}
\enquote{
Mittlerweile ist Objektorientierung so populär geworden, dass sich viele Software-Produkte, Werkzeuge und Vorgehensmodelle schon aus Marketing-Gründen objektorientiert nennen – unnötig zu sagen, dass nicht überall, wo ”objektorientiert“ draufsteht, auch ”objektorientiert“ drin ist.} 
\cite[S.16]{PoetzschHeffter.2009}
\end{quote}

Doch welche Merkmal müssen auf eine Programmiersprache zutreffen, um sie als objektorientierte Programmiersprache betiteln zu können?
Laut \cite{Lahres.2011} gelten folgende Grundelemente als Basis einer objektorientierten Programmiersprache:

\begin{itemize}
    \item Vererbung
    \item Datenkapselung
    \item Polymorphie
\end{itemize}

Erfüllen Go und Swift die Grundvoraussetzungen einer objektorientierten Programmiersprache? Dies soll in diesem Kapitel erörtert werden.

\section{Vererbung}
Ein zentrales Element der Objektorientierten Programmierung ist die Vererbung. 
Laut \cite[S. 145]{PoetzschHeffter.2009} bedeutet Vererbung im engeren Sinne, dass eine Klasse Programmteile von einer anderne Klasse automatisch übernimmt. 
Bei der einfachen Form der Vererbung, auch Einfachvererbung genannt, erbt eine Klasse von einer anderen Klasse.
Eine andere Form der Vererbung ist die sogenannte Merfachvererbung. 
Bei der Mehrfachvererbung kann eine Klasse von mehreren Klassen erben.
Von \cite[S.41]{Oestereich.1999} wird darauf hingewiesen, dass zu Vererbung auch verschieden Alternativen existieren und die Möglichkeiten und die Sinnhaftigkeit von Vererbung häufig überschätzt werden.


Dieser Abschnitt beschäftigt sich mit der Frage ob Go und Swift Vererbung unterstützen beziehungsweise ob eine Aufgabe alternativ implementiert werden kann.
Zu diesem Zweck soll eine gegebene Vererbungshierarchie in Form eines UML-Klassendiagramms beispielhaft in Go und Swift implementiert werden. 
Das folgende Beispiel orientiert sich an \cite[]{WilliamKennedy.2013}.


\section{Datenkapselung}




\section{Polymorphismus}