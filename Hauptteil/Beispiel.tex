\chapter*{Beispiele}
\section*{Code-Test}
\blindtext

    \begin{listing}[]
    \caption{C\# Hello World}
    \label{listing:helloWorld}
    \begin{minted}[]{csharp}
    // Hello1.cs
    public class Hello1
    {
       public static void Main()
       {
          System.Console.WriteLine("Hello, World!");
       }
    }
    \end{minted}
    \end{listing}

\blindtext
\newpage

\section*{Glossar-Test}
Um einen Eintrag im Glossar zu erzeugen, muss der defninierte Glossar-Begriff z.B. \Gls{computer} aufgerufen werden

%Um eine Abkürzung zu verwenden wird \acrlong{gcd} benutzt.
Um eine Abkürzung zu verwenden wird \gls{gcd} benutzt.
\newpage


\section*{Bild-Test}
\blindtext

	\begin{figure}[h]
	\centering
	\includegraphics[width=0.5\textwidth]{Images/latex}
	\caption{Ein Beispielbild}%
	\label{figure:latex}%
	\end{figure}

Eingebunden.
\blindtext
\newpage

\section*{Tabelle-Test}
Hier kommt eine Tabelle.
Die Tabelle \ref{table:test} wird so referenziert.
\blindtext
 
\begin{table}[h]
\centering
\begin{tabular}{|c|c|c|c|} 
 \hline
 \rowcolor[gray]{0.75} \textbf{Col1} & \textbf{Col2} & \textbf{Col2} & \textbf{Col3} \\
 \hline
 1 & 6 & 87837 & 787 \\ 
 \hline 
 2 & 7 & 78 & 5415 \\
 \hline 
 3 & 545 & 778 & 7507 \\
 \hline
 4 & 545 & 18744 & 7560 \\
 \hline
 5 & 88 & 788 & 6344 \\
 \hline
\end{tabular}
\caption{Table to test captions and labels}
\label{table:test}
\end{table}

\blindtext

\newpage

