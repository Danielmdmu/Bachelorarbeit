\chapter{Anwendungsbeispiel}
Dieses Kapitel beschäftigt sich damit, die Machbarkeit der Implementierung eines konkreten Anwendungsbeispiels zu überprüfen.
Als Anwendungsbeispiel soll ein ,,Hallo Welt''-Beispiel in Form einer einfachen Webanwendung umgesetzt werden.
Zur Umsetzung soll ein Web-Framework eingesetzt werden.

% Dazu soll zuerst ein Überblick über die verfügbaren Frameworks gegeben werden und anschließend die beispielhafte Implementierung eines ,,Hallo Welt''-Beispiels in einem ausgewählten Framework für Go und Swift vorgestellt werden.

% \section{Frameworks}
% Die beiden Aufstellungen von Web-Frameworks für Go und Swift stellen lediglich einen Auszug aus den vorhandenen Frameworks dar.
% Das Augenmerk 

% Go:
% \begin{itemize}
%     \item Revel \cite{Revel} - 8326
%     \item Beego \cite{Beego} - 11105
%     \item Gin Gonic \cite{GinGonic} 10190
%     \item Iris \cite{Iris} 6946
%     \item Echo \cite{Echo} 7507
% \end{itemize}

% Swift:
% \begin{itemize}
%     \item Kitura \cite{Kitura} 5727
%     \item Perfect \cite{Perfect} 11601
%     \item Vapor \cite{Vapor} 9799
%     \item Express \cite{Express} 8534
% \end{itemize}

Zur Implementierung wurde für Go das Framework \textit{Beego}\cite{Beego} ausgewählt.
Das Framework \textit{Beego} wurde ausgewählt, da es auf Github die meisten Follower hat.
Um mit \textit{Beego} zu arbeiten, müssen die Quelltext-Dateien des Framworks in den eigenen \textit{Go-Workspace} heruntergeladen werden.
\autoref{lst:BeegoInstallation} zeigt in Zeile 1 den Befehl zur Installation von \textit{Beego}.

\begin{listing}[H]
\caption{Installation von \textit{Beego} Quelle:\cite{Beego}}
\label{lst:BeegoInstallation}
\begin{Commandline}
go get -u github.com/astaxie/beego
\end{Commandline}
\end{listing}

In Zeile 3 von \autoref{lst:BeegoInstallation} wird im Go-Workspace ein neuer Ordner angelegt und in diesen Ordner gewechselt.
\autoref{lst:HelloBeego} zeigt die die Datei \textit{hello.go}. 
Es wird ein neuer \textit{MainController} erstellt, der auf eine \textit{Get}-Anfrage ,,hello world'' ausgibt.

\begin{listing}[H]
\caption{Hello World in Beego Quelle:\cite{Beego}}
\label{lst:HelloBeego}
\begin{GoCode}
package main

import (
    "github.com/astaxie/beego"
)

type MainController struct {
    beego.Controller
}

func (this *MainController) Get() {
    this.Ctx.WriteString("hello world")
}

func main() {
    beego.Router("/", &MainController{})
    beego.Run()
}
\end{GoCode}
\end{listing}

% \begin{listing}[H]
% \caption{Kompilieren und starten Quelle:\cite{Beego}}
% \label{lst:BeegoBuild}
% \begin{Commandline}
% go build -o hello hello.go

% ./hello
% \end{Commandline}
% \end{listing}

Als Web-Framework für Swift wurde \textit{Kitura} ausgewählt.
\textit{Kitura} wurde ausgewählt, da es von IBM entwickelt wird.

\begin{listing}[H]
\caption{Swift-Projekt erstellen Quelle:\cite{Kitura}}
\label{lst:KituraInstallation}
\begin{Commandline}
mkdir myFirstProject
cd myFirstProject
swift package init --type executable
\end{Commandline}
\end{listing}

Die Einbindung von \textit{Kitura} erfolgt über die vom \textit{Package Manager} erstellte Datei \textit{Package.swift}.

\begin{listing}[H]
\caption{\textit{Kitura} einbinden über \textit{Package.swift} Quelle:\cite{Kitura}}
\label{lst:KituraInstallation2}
\begin{SwiftCode}
import PackageDescription

let package = Package(
    name: "myFirstProject",
    dependencies: [
        .Package(
            url: "https://github.com/IBM-Swift/Kitura.git", 
            majorVersion: 1, 
            minor: 7
        )
    ])
\end{SwiftCode}
\end{listing}

\begin{listing}[H]
\caption{\textit{main.swift} Quelle:\cite{Kitura}}
\label{}
\begin{SwiftCode}
import Kitura

// Create a new router
let router = Router()

// Handle HTTP GET requests to /
router.get("/") {
    request, response, next in
    response.send("Hello, World!")
    next()
}

// Add an HTTP server and connect it to the router
Kitura.addHTTPServer(onPort: 8080, with: router)

// Start the Kitura runloop (this call never returns)
Kitura.run()
\end{SwiftCode}
\end{listing}

% \begin{listing}[H]
% \caption{Kompilieren und starten Quelle:\cite{Kitura}}
% \label{lst:KituraBuild}
% \begin{Commandline}
% swift build

% .build/debug/myFirstProject
% \end{Commandline}
% \end{listing}