\chapter{Generische Programmierung}
Generische Programmierung erlaubt es, flexible und wiederverwendbare Funktionen zu schreiben. 
Sie ermöglicht es, Duplikation zu vermeiden und verständlichen Quelltext zu entwickeln \cite[S.371]{Apple.2017}.


Die FAQ von Go ist zu entnehmen, das Go derzeit keine generische Programmierung unterstützt. 
Jedoch ist es möglich, dass Mechanismen zur generischen Programmierung im Laufe der Zeit in Go integriert werden \cite{Golang.FAQ}.


Swift bietet direkte Untestützung für generische Programmierung.
Laut \cite[S.371]{Apple.2017} sind Generics eine der mächtigsten Funktionen von Swift und werden auch in der Standardbibliothek von Swift ausgiebig genutzt.
Programmierer, welche beispielsweise aus dem Java-Umfeld kommen, sollten laut \cite[S.206]{Hoffman.2017} keine Probleme haben, sich mit generischer Programmierung in Swift zurecht zu finden.
\autoref{lst:GenericFunctionSwift} zeigt ein Beispiel für eine generische Funktion und \autoref{lst:GenericTypeSwift} ein Beispiel für eine generische Klasse.

\begin{listing}[H]
\caption{Generische Klasse in Swift \\ Quelle:\cite[S.213]{Hoffman.2017}}
\label{lst:GenericTypeSwift}
\begin{SwiftCode}
class List<T> {
    var items = [T]()
    
    func add(item: T) {
        items.append(item)
    }
    
    func getItemAtIndex(index: Int) -> T? {
        if items.count > index {
            return items[index]
        } else {
            return nil
        }
    }
}
\end{SwiftCode}
\end{listing}

\begin{listing}[H]
\caption{Generische Funktion in Swift \\ Quelle:\cite[S.373]{Apple.2017}}
\label{lst:GenericFunctionSwift}
\begin{SwiftCode}
func swapTwoValues<T>(_ a: inout T, _ b: inout T) {
    let temporaryA = a
    a = b
    b = temporaryA
}
\end{SwiftCode}
\end{listing}


