\chapter{Zusammenfassung}
Go und Swift bieten einige interessante Möglichkeiten Software zu entwickeln.
Durch eine moderne Syntax, welche beispielsweise auf den Einsatz von Semikolons verzichtet,
erleichtern beide Sprachen die Entwicklung und Lesbarkeit des Quelltextes.
Der Verzicht von Semikolons in Go bedeutet für den Programmierer allerdings auch, sich an andere Syntax-Regeln halten zu müssen.

Beide Sprachen besitzen ein statisches Typsystem.
Bei der Entwicklung für 32- und 64-Bit Systeme muss allerdings auf die richtige Typisierung von Variablen und Konstanten geachtet werden, um eine Lauffähigkeit auf beiden Plattformen ohne mögliche Überschreitungen der Wertebereiche, zu gewährleisten.
Dies betrifft vor allem die Verwendung von \textit{Type Inference}.

Grundsätzlich ist objektorientierte Programmierung in Go und Swift möglich.
Umsteiger aus Java wird allerdings der Umstieg auf Swift leichter fallen als in Go.
Da Go auf Vererbung verzichtet, in Swift ist Vererbung integriert.
Swift bietet im Gegensatz zu Go integrierte Getter/Setter-Methoden.
Die Möglichkeiten zur Zugriffssteuerung sind in Swift besser als in Go.
Mit der Fähigkeit durch \textit{Extensions} eine Klasse/Objekt durch neue Eigenschaften beziehungsweise Verhalten zu erweitern, hebt sich Swift deutlich von Go ab.

Die Art der Fehlerbehandlung unterscheidet sich in Go und Swift grundlegend, da Go keine \textit{Exceptions} unterstützt.
Swift bietet ausgeprägte Mechanismen zur Fehlerbehandlung.

Swift ist aktuell noch nicht für Windows verfügbar. 
Go hingegen ist für Windows, Linux und MacOS verfügbar.
Beim Einsatz einer \gls{IDE} für Swift ist man derzeit auf XCode, welches nur für MacOS verfügbar ist, beschränkt.
Für Go ist mit \textit{LiteIDE} eine \gls{IDE} für Windows, Linux und MacOS verfügbar.

Beide Sprachen setzen zu Speicherverwaltung einen \textit{Garbage Collector} ein.
Das Thema der Speicherverwaltung spielt besonders hinsichtlich der nebenläufigen Programmierung eine große Rolle.

Go und Swift bieten dem Programmierer gute Konzepte zur nebenläufigen Programmierung. 
Beide Sprachen erleichtern es dem Programmierer Programmteile nebenläufig ausführen zu lassen. 

Mit Go kann nicht generisch programmiert werden.
Swift hingegen unterstützt generische Programmierung und sollte Programmierern, die Java beherrschen, keine Probleme beim Umstieg bereiten.

Obwohl Go und Swift nicht explizit funktionale Programmiersprachen sind, ist es möglich mit ihnen funktional zu programmieren, was dem Programmierer interessante Möglichkeiten bietet.

Go präsentiert sich als leichtgewichtige Sprache, welche besonders für Systemprogrammierung und serverseitige Programmierung geeignet scheint.
Swift hingegen versucht sich als Allgemein-Programmiersprache zu etablieren.
Besonders in der objektorientierten Programmierung bietet Swift Vorteile.

\chapter{Ausblick}
Beide Sprachen werden sich in den nächsten Jahren weiterentwickeln und an Bedeutung gewinnen. 
Eine Implementierung von Swift für Windows ist derzeit noch nicht verfügbar, könnte aber die Bekanntheit von Swift steigern.
Da beide Sprachen Open-Source sind, wird es auch interessant werden, wie stark die Community der beiden Sprachen wächst und Entwicklungen vorantreibt.

Beim Thema objektorientierte Programmierung wäre die Implementierung eines größeren Beispiels aus der Realität von Interesse.
Besonders in Betracht der fehlenden Vererbungsmechanismen in Go und ob sich dieser Aspekt negativ bei der Implementierung auswirkt.

Hinsichtlich nebenläufiger Programmierung wäre ein genauer Vergleich der Synchronisierungsmechanismen interessant.