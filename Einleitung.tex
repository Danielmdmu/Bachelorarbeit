\chapter{Einleitung}

\section{Vorstellung des Kontexts der Arbeit}
Neben den etablierten Programmiersprachen, wie Java und C/C++, werden kontinuierlich neue Programmiersprachen entworfen.
Viele neue Sprachen werden als Open-Source entwickelt und auf Github gehostet.

Folgende Auflistung zeigt die auf Github beliebtesten Programmiersprachen-Projekte (Stand: Mai 2017) \cite{Github}.

% \begin{table}[H]
%     \centering
%     \begin{tabularx}{\textwidth}{|l|X|}
%     \hline 
%     \rowcolor[gray]{0.75} \textbf{Platz} & \textbf{Programmiersprache} \\
%     \hline
% 	1 & Swift \\
% 	\hline
% 	2 & Go \\
% 	\hline
% 	3 & TypeScript \\
% 	\hline  
% 	4 & Rust \\
% 	\hline
% 	5 & Kotlin \\
%     \hline
%     \end{tabularx}
%     \caption{Rangfolge der Programmiersprachen Projekte auf Github, in Anlehung an \cite{Github}}
%     \label{tab:Github}
% \end{table}

\begin{itemize}
    \item Swift
    \item Go
    \item TypeScript
    \item Rust
    \item Kotlin
\end{itemize}

Mit Go und Swift befinden sich zwei Projekte darunter, die aufgrund ihrer Herkunft besondere Aufmerksamkeit erregen.
Go wird seit 2009 von Google entwickelt und Swift wurde 2014 von Apple vorgestellt.
Doch warum sind Google und Apple daran interessiert neue Programmiersprachen zu entwickeln?
Die \gls{FAQ} von Go liefert auf die Frage ,,Why are you creating a new language?'' folgende Antwort: 

\begin{quote}
\enquote{Go was born out of frustration with existing languages and environments for systems programming.}\cite{Golang.FAQ}
\end{quote}

Rob Pike, einer der Entwickler von Go, leitet einen seiner Artikel zum Einsatz von Go bei Google, mit folgender Aussage ein:

\begin{quote}
\enquote{Go was designed to address the problems faced in software development at Google, which led to a language that is not a breakthrough research language but is nonetheless an excellent tool for engineering large software projects.}\cite{RobPike.2012}
\end{quote}

Google entwickelt Go also vor allem aus eigenem Bedarf heraus und um bestehende Probleme bei Google zu lösen.
Auf die Frage, warum Swift entwickelt wurde, gibt die offizielle Webseite von Swift folgende Antwort:

\begin{quote}
\enquote{The goal of the Swift project is to create the best available language for uses ranging from systems programming, to mobile and desktop apps, scaling up to cloud services.}\cite{Swift.Homepage}
\end{quote}

Swift verfolgt mit dem Ziel, eine der besten verfügbaren Programmiersprachen zu entwickeln, ein sehr ambitioniertes Ziel.

\section{Motivation}
Google und Apple verfolgen demnach beide das Ziel die heutigen Anforderungen an moderne Softwareentwicklung mit einer neuen Programmiersprache zu erfüllen.  
Im April 2016 war im Onlinemagazin heise.de folgende Schlagzeile zu lesen:

\begin{quote}
\enquote{Google erwägt Swift angeblich als Programmiersprache für Android}\cite{Becker}
\end{quote}

% Doch warum erwägt Google Swift als Programmiersprache für Android, wenn von Google mit Go eine eigene moderne Programmiersprache entwickelt wird?
Obwohl Google mit Go selbst eine moderne Programmiersprache entwickelt, ist Google an einem Einsatz von Swift interessiert.
Anfang 2017 war zudem auf heise.de diese Schlagzeile zu lesen:

\begin{quote}
\enquote{Programmiersprachen-Ranking: TIOBE sieht Go als Sprache des Jahres 2016}\cite{MengeSonnentag}
\end{quote}

\autoref{tab:Tiobe} zeigt die Top 20 des TIOBE-Index zum Stand Mai 2017. 
Go befindet sich auf Platz 16 und Swift belegt Platz 13. 
In der zweiten Spalte ist die Platzierung des Vorjahres zu sehen. 
Go hat sich dabei von Platz 42 auf Platz 16 verbessert.

\begin{table}[H]
    \centering
    \begin{tabularx}{\textwidth}{|c|c|X|}
    \hline 
    \rowcolor[gray]{0.75} \textbf{Mai 2017} & \textbf{Mai 2016} & \textbf{Programmiersprache} \\
    \hline
    1 & 1 & Java \\
    \hline
    2 & 2 & C \\
    \hline
    3 & 3 & C++ \\
    \hline
    4 & 5 & Python \\
    \hline
    5 & 4 & C\# \\
    \hline
    6 & 10 & Visual Basic.NET \\
    \hline
    7 & 7 & JavaScript \\
    \hline
    8 & 12 & Assembler \\
    \hline
    9 & 6 & PHP \\
    \hline
    10 & 9 & Perl \\
    \hline
    11 & 8 & Ruby \\
    \hline
    12 & 13 & Visual Basic \\
    \hline
    \rowcolor[gray]{0.9} 13 & 15 & Swift \\
    \hline
    14 & 16 & R \\
    \hline
    15 & 14 & Objective-C \\
    \hline
    \rowcolor[gray]{0.9} 16 & 42 & Go \\
    \hline
    17 & 18 & MATLAB \\	
    \hline
    18 & 11 & Delphi/Object Pascal \\	
    \hline
    19 & 19 & PL/SQL \\	
    \hline
    20 & 22 & Scratch \\	
    \hline
    \end{tabularx}
    \caption{TIOBE-Index Stand Mai 2017, in Anlehung an \cite{Tiobe}}
    \label{tab:Tiobe}
\end{table}

\section{Aufgabenstellung}
Go und Swift haben somit Interesse bei Softwareentwicklern geweckt.
In \autoref{tab:Vergleich} ist eine Gegenüberstellung von Go und Swift zu sehen.

\begin{table}[H]
    \centering
    \begin{tabularx}{\textwidth}{|X|X|}
    \hline 
    \rowcolor[gray]{0.75} \textbf{Go} & \textbf{Swift} \\
    \hline
	Google & Apple \\
	\hline
	Open Source & Open Source \\
	\hline
	Kompiliert & Kompiliert \\
	\hline  
	Statisches Typsystem & Statisches Typsystem \\
	\hline
	Garbage Collected & Garbage Collected \\
    \hline
    \end{tabularx}
    \caption{Gegenüberstellung Go und Swift}
    \label{tab:Vergleich}
\end{table}

% Der einzige Unterschied der beiden Sprachen, bei diesem Kriterien, ist, dass Go von Google und Swift von Apple entwickelt wurde. 
Ein oberflächlicher Vergleich von Go und Swift wird zu dem Ergebnis kommen, dass die beiden Sprachen gleich sind.
%Doch worin liegen die Unterschiede der beiden Programmiersprachen?
Die Unterschiede von Go und Swift müssen also im Detail liegen.
%Welche Möglichkeiten gibt es, Programmiersprachen zu vergleichen?
Ein näherer Vergleich der beiden Programmiersprachen sollte mögliche Gemeinsamkeiten und Unterschiede aufzeigen können.

Von \cite{Comparsion} wurde ein Vergleich von Programmiersprachen durchgeführt.
Einige der Vergleichskriterien dieser Arbeit sind:

\begin{itemize}
    \item Speicherverbrauch
    \item Kompilierzeit
    \item Laufzeit
    \item Größe des erzeugten Binärcodes
\end{itemize}

% Diese aufgeführten Vergleichskriterien sind zwar messbar, allerdings spiegeln sie nicht die Anforderungen an moderne Programmiersprachen wieder. 
Die aufgeführten Kriterien sind zwar messbar, allerdings sind diese Kriterien sehr abhängig von der zu Grunde liegenden Hardware.

Aus diesem Grund wurden für diese Arbeit eigene Vergleichskriterien aufgestellt, um Go und Swift miteinander zu vergleichen.
Die beiden Sprachen sollen mit diesen Kriterien miteinander verglichen werden.

\section{Aufbau der Arbeit}
Jedes Kapitel dieser Arbeit entspricht einem Vergleichskriterium.
Die zugehörigen Abschnitte konkretisieren das Vergleichskriterium.
Als nötiges Vorwissen wird vom Leser erwartet, dass er Kenntnisse in einer höheren Programmiersprache, wie Java, C\# oder C/C++, besitzt. 